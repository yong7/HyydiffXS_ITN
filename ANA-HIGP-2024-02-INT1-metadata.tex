%-------------------------------------------------------------------------------
% This file contains the title, author and abstract.
% It also contains all relevant document numbers used for an ATLAS note.
%-------------------------------------------------------------------------------

% Title
\AtlasTitle{Measurement of fiducial and differential cross sections in the $H\rightarrow\gamma\gamma$ decay channel with  \SI{168}{\ifb} of \SI{13.6}{TeV} proton–proton collision data with the ATLAS detector}

% Draft version:
% Should be 1.0 for the first circulation, and 2.0 for the second circulation.
% If given, adds draft version on front page, a 'DRAFT' box on top of each other page, 
% and line numbers.
% Comment or remove in final version.
\AtlasVersion{0.1}

% Abstract - % directly after { is important for correct indentation
\AtlasAbstract{%
  We present supporting documentation for fiducial and differential cross-section measurements of the Higgs boson in the $H\rightarrow\gamma\gamma$ decay channel using \SI{168}{\per\fb} of proton--proton collisions recorded at \SI{13.6}{TeV} centre of-mass energy in the years 2022 to 2024.
  %The amount of background, mainly from SM diphoton production and hadronic jets, is determined by a simultaneous signal and background fit to the diphoton mass spectrum.
  The fiducial cross-section is measured in few inclusive phase space regions, and differentially as functions of a selection of variables.
  %The data is compared to several state-of-the-art theoretical predictions of SM Higgs boson production.
}

% Author - this does not work with revtex (add it after \begin{document})
\author{The ATLAS Collaboration}

% Authors and list of contributors to the analysis
% \AtlasAuthorContributor also adds the name to the author list
% Include package latex/atlascontribute to use this
% Use authblk package if there are multiple authors, which is included by latex/atlascontribute
% \usepackage{authblk}
% Use the following 3 lines to have all institutes on one line
% \makeatletter
% \renewcommand\AB@affilsepx{, \protect\Affilfont}
% \makeatother
% \renewcommand\Authands{, } % avoid ``. and'' for last author
% \renewcommand\Affilfont{\itshape\small} % affiliation formatting
\AtlasAuthorContributor{First AtlasAuthorContributor}{a}{Author's contribution.}
\AtlasAuthorContributor{Second AtlasAuthorContributor}{b}{Author's contribution.}
\AtlasAuthorContributor{Third AtlasAuthorContributor}{a}{Author's contribution.}
%\AtlasContributor{Fourth AtlasContributor}{b}{Contribution to the analysis.}
%\author[a]{First Author}
% \author[a]{Second Author}
% \author[b]{Third Author}
\affil[a]{One Institution}
\affil[b]{Another Institution}


% If a special author list should be indicated via a link use the following code:
% Include the two lines below if you do not use atlasstyle:
% \usepackage[marginal,hang]{footmisc}
% \setlength{\footnotemargin}{0.5em}
% Use the following lines in all cases:
% \usepackage{authblk}
% \author{The ATLAS Collaboration%
% \thanks{The full author list can be found at:\newline
%   \url{https://atlas.web.cern.ch/Atlas/PUBNOTES/ATL-PHYS-PUB-2016-007/authorlist.pdf}}
% }

% ATLAS reference code, to help ATLAS members to locate the paper
\AtlasRefCode{ANA-HIGP-2024-02}

% ATLAS note number. Can be an COM, INT, PUB or CONF note
\AtlasNote{ANA-HIGP-2024-02-INT1}
