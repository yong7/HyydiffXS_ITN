%-------------------------------------------------------------------------------
% This file provides template EXOT group object descriptions and cuts.
\pdfinclusioncopyfonts=1
% This command may be needed in order to get \ell in PDF plots to appear. Found in
% https://tex.stackexchange.com/questions/322010/pdflatex-glyph-undefined-symbols-disappear-from-included-pdf
%-------------------------------------------------------------------------------
% Specify where ATLAS LaTeX style files can be found.
% As of TeX Live 2020 the slash is not needed after the directory name.
\makeatletter
\providecommand\IfFormatAtLeastTF{\@ifl@t@r\fmtversion}
\IfFormatAtLeastTF{2020/10/01}{%
  \def\input@path{{..}{../../latex}}
}{
  \def\input@path{{../}{../../latex/}}
}
\makeatother

%-------------------------------------------------------------------------------
\documentclass[NOTE, REPORT=false, atlasdraft=false, USenglish]{atlasdoc}
% The language of the document must be set: usually UKenglish or USenglish.
% british and american also work!
% Commonly used options:
%  atlasdraft=true|false This document is an ATLAS draft.
%  texlive=YYYY          Specify TeX Live version (2020 is default).
%  txfonts=true|false    Use txfonts rather than the default newtx
%  paper=a4|letter       Set paper size to A4 (default) or letter.

%-------------------------------------------------------------------------------
% Extra packages:
\usepackage[biblatex=false]{atlaspackage}
% Commonly used options:
%  subfigure|subfig|subcaption  to use one of these packages for figures in figures.
%-------------------------------------------------------------------------------
\usepackage{multirow}

% Useful macros
\usepackage[jetetmiss]{atlasphysics}
% See doc/atlas_physics.pdf for a list of the defined symbols.
% Default options are:
%   true:  journal, misc, particle, unit, xref
%   false: BSM, heppparticle, hepprocess, hion, jetetmiss, math, process,
%          other, snippets, texmf
% See the package for details on the options.

% Add your own definitions here (file atlas-document-defs.sty).
% \usepackage{atlas-document-defs}

% Paths for figures - do not forget the / at the end of the directory name.
\graphicspath{{../../logos/}{figures/}}

%-------------------------------------------------------------------------------
% Generic document information
%-------------------------------------------------------------------------------

\AtlasTitle{EXOT group text snippets for INT notes}

\author{ATLAS EXOT Group}

\AtlasRefCode{Version \ATPackageVersion}

\AtlasAbstract{%
  This note contains text snippets and tables that should be included in supporting notes
  from the EXOT group.

  The templates are in American English.
  If wanted, some adaption to British English could be made. 

  % \emph{2018-10-23: This file is a work in progress (WIP) and will probably be updated.
  % Backwards incompatible changes may be made as the examples develop.}
}
% Author and title for the PDF file
\hypersetup{pdftitle={ATLAS EXOT supporting note},pdfauthor={ATLAS EXOT group}}

%-------------------------------------------------------------------------------
% Main document
%-------------------------------------------------------------------------------
\begin{document}

\maketitle

\tableofcontents

\section{Introduction}
\textit{This is a very brief, almost ``abstract-like'' section.  Immediately following is the 
Executive Summary, which should include all of the components that are sometimes in an 
introduction but they are organized in a way that will facilitate review by conveners
since they are in a standard way}


\section{Executive Summary}
\textit{This section, ideally 2-pages (max), should be placed at the beginning of the internal 
note following the more conventional introduction.  It should be split as highlighted here and should 
give a high-level overview of the analysis including (but not limited to):}

\begin{itemize}
\item \textit{Motivation, physics target, and the general characteristics of the signal}
\item \textit{Analysis strategy}
\item \textit{General characteristics of the control, validation, and signal regions}
\item \textit{Background estimation strategy overview}
\item \textit{Highlight major or most important points of the analysis}
\item \textit{Team overview task list including a list of all critical tasks, who is responsible for each task, and what else they are working on outside of this analysis.  This should be presented in the format shown in Table~\ref{tab:Miles_Ahead}.}
\item \textit{List of outstanding items in the analysis that still need to be addressed}
\end{itemize}

\include{executive_summary}

\section{Data and MC}
\textit{Dataset used with blinding strategy, full list of background samples and details of the signal samples.}


\section{Object selection}
\textit{The supporting notes should now include the following standardized tables of properties: 
each analysis should simply fill them in by writing / replacing the value with the appropriate 
number or by choosing the appropriate option. The idea of these tables is to harmonize some sections 
of the supporting notes as to make review and analysis comparisons simpler.}

\textit{If you use non-standard selections which do not fit in these tables, this should of course 
be noted and discussed in more detail in the text.}

\textit{Object selection tables (following template) and detailed event selection: there may, of 
course, still be some minor open items, as long as they don’t significantly affect the analysis 
strategy, but these should be well defined and clearly indicated (e.g. coloured/bold) in the text 
in this section and in the list of outstanding tasks within the executive summary. Both should 
be updated as the analysis progresses.}
 
\subsection{Electron selection}

\begin{table}[ht]
  \caption{Electron selection criteria.}%
  \label{tab:object:electron}
  \centering
  % \resizebox{\textwidth}{!}{
  \begin{tabular}{ll}
    \toprule
    Feature & \multicolumn{1}{c}{Criterion} \\
    \midrule
    Pseudorapidity range & \(|\eta| <\) X\\
    Energy calibration & \texttt{es2017\_R21\_PRE} (ESModel)\\
    Energy & \(E > \qty[parse-numbers=false]{XX}{\GeV}\) \\
    Transverse energy & \(\ET > \qty[parse-numbers=false]{XX}{\GeV}\) \\
    Transverse momentum & \(\pT > \qty[parse-numbers=false]{XX}{\GeV}\) \\
    \midrule
    \multirow{2}{*}{Object quality} & Not from a bad calorimeter cluster (\texttt{BADCLUSELECTRON})\\ %\cline{2-2}
      & Remove clusters from regions with EMEC bad HV (2016 data only) \\
    \midrule
    \multirow{2}{*}{Track to vertex association} & \(|d_{0}^{\text{BL}}(\sigma)| < X\) \\ %\cline{2-2}
    & \(|\Delta z_{0}^{\text{BL}} \sin{\theta}| < \qty[parse-numbers=false]{X}{\mm}\) \\
    \midrule
    Identification & (\texttt{Loose/Medium/Tight}) \\
    Isolation & \texttt{LooseTrackOnly / Loose / Tight / Gradient / \ldots} \\
      \bottomrule
  \end{tabular}
  % }
\end{table}

Notes:
\begin{itemize}
\item Pseudorapidity: when the calorimeter crack is not excluded, the range can be indicated simply as \enquote{\(|\eta| < 2.47\)}, when the crack is excluded: \enquote{\((|\eta| < 1.37) \quad || \quad (1.52 < |\eta| < 2.47)\)}.
\item Usually only one among \enquote{Energy}, \enquote{Transverse energy} and \enquote{Transverse momentum} criteria is applied --- the \qty{30}{\GeV} value is just an example.
  In special cases energy (i.e.\ calorimeter-based measurement) and momentum (i.e.\ tracking-based measurement) criteria can be required in order to constraint different aspects of the reconstruction.
\item Electron ID\@: 3 working points (Loose/Medium/Tight) are evaluated using the Likelihood-based (LH) method, by the
  \href{https://twiki.cern.ch/twiki/bin/view/AtlasProtected/EGammaIdentificationRun2}{ElectronPhotonSelectorTools}.
\item Energy calibration of electrons is implemented in the\\
  \href{https://twiki.cern.ch/twiki/bin/view/AtlasProtected/ElectronPhotonFourMomentumCorrection}{ElectronPhotonFourMomentumCorrection} tool.
\item Scale Factors for efficiencies for electrons are implemented in the\\
  \href{https://twiki.cern.ch/twiki/bin/view/AtlasProtected/XAODElectronEfficiencyCorrectionTool}{ElectronEfficiencyCorrection} tool.
\item Updated configurations for the EGamma CP tools can be found on this \href{https://twiki.cern.ch/twiki/bin/view/AtlasProtected/EGammaRecommendationsR21}{TWiki} page.
\end{itemize}

\newpage

\subsection{Photon selection}

\begin{table}[ht]
  \caption{Photon selection criteria.}%
  \label{tab:object:photon} 
  \centering
  % \resizebox{\textwidth}{!}{
  \begin{tabular}{ll}
    \toprule
    Feature & \multicolumn{1}{c}{Criterion} \\
    \midrule
    Pseudorapidity range & \(|\eta| <\) X\\
    Energy calibration & \texttt{es2017\_R21\_PRE} (ESModel)\\
    Energy & \(E > \qty[parse-numbers=false]{XX}{\GeV}\) \\
    Transverse energy & \(\ET > \qty[parse-numbers=false]{XX}{\GeV}\) \\
    \midrule
    \multirow{2}{*}{Object quality} & Not from a bad calorimeter cluster (\texttt{BADCLUSELECTRON})\\ %\cline{2-2}
      & Remove clusters from regions with EMEC bad HV (2016 data only) \\
    \midrule
    Photon cleaning & \texttt{passOQquality} \\
    Fudging & Applied for Full sim / not for AtlFastII \\
    \midrule
    Identification & (\texttt{Loose/Tight}) \\
    Isolation &  \texttt{FixedCutTightCaloOnly / FixedCutTight / FixedCutLoose} \\
    \bottomrule
  \end{tabular}
  %  }
\end{table}

Notes:
\begin{itemize}
\item Pseudorapidity: please note that the maximum value for \(|\eta|\) for photon candidates (2.37) is smaller than for electron candidates (2.47). 
  If crack excluded: \enquote{\((|\eta| < 1.37) \quad || \quad (1.52 < |\eta| < 2.37)\)}.
\item Usually only one between \enquote{Energy} and \enquote{Transverse energy} criteria is applied --- the \qty{30}{\GeV} value is just an example.
\item Photon cleaning: a new Photon helper is available to apply the photon cleaning cut 
  (from the \texttt{ElectronPhotonSelectorTools}, tag \(\ge\) 00-02-92-21, release \(\ge\) 2.4.30).
\item Photon ID\@: 2 working points (Loose/Tight) are evaluated using a cut-based method, by the
  \href{https://twiki.cern.ch/twiki/bin/view/AtlasProtected/EGammaIdentificationRun2}{ElectronPhotonSelectorTools}.
\item Energy calibration of photons is implemented in the\\
  \href{https://twiki.cern.ch/twiki/bin/view/AtlasProtected/ElectronPhotonFourMomentumCorrection}{ElectronPhotonFourMomentumCorrection} tool.
\item Scale Factors for efficiencies for photons are implemented in the\\
  \href{https://twiki.cern.ch/twiki/bin/view/AtlasProtected/XAODElectronEfficiencyCorrectionTool}{ElectronEfficiencyCorrection} tool.
\item Updated configurations for the EGamma CP tools can be found on this \href{https://twiki.cern.ch/twiki/bin/view/AtlasProtected/EGammaRecommendationsR21}{TWiki} page.
\end{itemize}

\subsection{Muon selection}

\begin{table}[ht]
  \caption{Muon selection criteria.}%
  \label{tab:object:muon}
  \centering
  % \resizebox{\textwidth}{!}{
  \begin{tabular}[ht]{ll}
    \toprule
    Feature & Criterion \\
    \midrule
    Selection working point & \texttt{Loose/Medium/Tight /High-pT} \\
    Isolation working point & \texttt{LooseTrackOnly/Loose/Tight/Gradient/\ldots}\\
    Momentum calibration & Sagitta correction [used/not used] \\
    \pT Cut & \qty[parse-numbers=false]{X}{\GeV} \\
    \(|\eta|\) cut & \(< X\) \\
    \dzero significance cut & X \\
    \(z_{0}\) cut & \qty[parse-numbers=false]{X}{\mm} \\
    \bottomrule
  \end{tabular}
  % }
\end{table}

The selection criteria are implemented in the \texttt{MuonSelectorTools-XX-XX-XX}\\
with \texttt{MuonMomentumCorrections-XX-XX-XX}, 
isolation in \texttt{IsolationSelection-XX-XX-XX} and \dzero and \(z_{0}\) cuts in \texttt{xAODTracking-XX-XX-XX}.
The muon recommendations can be found in 
\href{https://twiki.cern.ch/twiki/bin/view/AtlasProtected/MCPAnalysisGuidelinesMC16}{MCPAnalysisGuidelinesMC16}.

\subsection{Tau selection}

\begin{table}[ht]
  \caption{Tau selection criteria.}%
  \label{tab:object:tau}
  \centering
  % \resizebox{\textwidth}{!}{
  \begin{tabular}{ll}
  \toprule
  Feature & Criterion \\
  \midrule
  Pseudorapidity range & \(|\eta| < X\) \\
  Track selection & 1 or 3 tracks \\\
  Charge & \(|Q| = 1\) \\
  Tau energy scale & \texttt{MVA TES}\\
  Transverse momentum & \(\pT > \qty[parse-numbers=false]{XX}{\GeV}\) \\
  Jet rejection & BDT-based (\texttt{Loose/Medium/Tight}) \\
  Electron rejection & BDT-based\\
  Muon rejection & Via overlap removal in \(\Delta R < 0.2\) and \(\pT > \qty{2}{\GeV}\).
    Muons must not be Calo-tagged\\
  \bottomrule
  \end{tabular}
  % }
\end{table}

If the crack is excluded: \((|\eta| < 1.37) || (1.52 < |\eta| < 2.5)\)

The selection criteria are all implemented in the \texttt{TauSelectionTool} as part of the \texttt{TauAnalysisTools}.
Documentation can be found in the \href{https://gitlab.cern.ch/atlas/athena/blob/21.2/PhysicsAnalysis/TauID/TauAnalysisTools/doc/README-TauSelectionTool.rst}{README-TauSelectionTool.rst}.

\subsection{Small-$R$ jet selection}

If you want to use variables such as \verb|\fcut| you need to add the option
\texttt{jetetmiss} to \texttt{atlaspackage}.

\begin{table}[ht]
  \caption{Jet reconstruction criteria.}%
  \label{tab:object:jet1}
  \centering
  % \resizebox{\textwidth}{!}{
  \begin{tabular}{ll}
  \toprule
  Feature & Criterion \\
  \midrule
  Algorithm & \Antikt  \\
  \(R\)-parameter & 0.4 \\
  Input constituent & EMTopo \\
  Analysis release number & 21.2.10 \\
  %Calibration tag & JetCalibTools-00-04-76 \\
  \texttt{CalibArea} tag & 00-04-81 \\
  Calibration configuration & \texttt{JES\_data2017\_2016\_2015\_Recommendation\_Feb2018\_rel21.config} \\
  Calibration sequence (Data) & \texttt{JetArea\_Residual\_EtaJES\_GSC\_Insitu} \\
  Calibration sequence (MC) & \texttt{JetArea\_Residual\_EtaJES\_GSC} \\
  %Calibration sequence (AFII) & \texttt{JetArea\_Residual\_EtaJES\_GSC} \\
  \midrule
  \multicolumn{2}{c}{Selection requirements} \\
  \midrule
  Observable & Requirement \\
  \midrule
  Jet cleaning & \texttt{LooseBad} \\
  BatMan cleaning & No \\
  \pT & \(> \qty[parse-numbers=false]{XX}{\GeV}\) \\
  \(|\eta|\) & \(< X\) \\
  JVT & (\emph{Update if needed}) \(>0.59\) for \(\pT < \qty{60}{\GeV}\), \(|\eta| < 0.4\)\\
  \bottomrule
  \end{tabular}
  % }
\end{table}


\clearpage
\subsection{Large-$R$ jet selection}

\begin{table}[ht]
  \caption{Large-\(R\) jet reconstruction criteria.}%
  \label{tab:object:jet2}
  \centering
  % \resizebox{\textwidth}{!}{
  \begin{tabular}{ll}
    \toprule
    Feature & Criterion \\ 
    \midrule
    Algorithm & \antikt  \\
    R-parameter & 1.0 \\
    Input constituent & \texttt{LCTopo} \\
    Grooming algorithm & Trimming \\ 
    \fcut & 0.05 \\
    \(R_{\text{trim}}\) & 0.2 \\
    Analysis release number & 21.2.10 \\
    %Calibration tag & JetCalibTools-00-04-76 \\
    \texttt{CalibArea} tag & 00-04-81 \\
    Calibration configuration & \texttt{JES\_MC16recommendation\_FatJet\_JMS\_comb\_19Jan2018.config} \\
    Calibration sequence (Data) & \texttt{EtaJES\_JMS\_Insitu} \\
    Calibration sequence (MC) & \texttt{EtaJES\_JMS} \\
    \bottomrule
    \multicolumn{2}{c}{Selection requirements} \\
    \midrule
    Observable & Requirement \\
    \midrule
    \pT  & \(> \qty[parse-numbers=false]{XX}{\GeV}\) \\
    \(|\eta|\) & \(< X\) \\
    Mass & \(> \qty[parse-numbers=false]{XX}{\GeV}\) \\
    \bottomrule
    \multicolumn{2}{c}{Boosted object tagger} \\
    \midrule
    Object  & Working point \\
    \midrule
    \(W\) / \(Z\) / top & 50\% / 80\% \\
    \(X\rightarrow bb\) & single/double \btag with/without loose/tight mass \\
    \bottomrule
  \end{tabular}
  % }
\end{table}


\subsection{\MET selection}

\begin{table}[ht]
  \caption{\MET reconstruction criteria.}%
  \label{tab:object:met}
  \centering
  \begin{tabular}{ll}
    \toprule
    Parameter & Value \\ 
    \midrule
    Algorithm & Calo-based \\
    Soft term & Track-based (TST) \\ 
    MET operating point & \texttt{Tight} \\
    Analysis release & 21.2.16 \\
    Calibration tag & \texttt{METUtilities-00-02-46} \\
    \bottomrule
    \multicolumn{2}{c}{Selection requirements} \\
    \midrule
    Observable & Requirement \\
    \midrule
    \MET & \(> \qty[parse-numbers=false]{XX}{\GeV}\) \\
    \(\sum{\ET} / \MET\)  & \(< X\) \\
    Object-based \MET significance & \(> X\) \\
    \bottomrule
  \end{tabular}
\end{table}



\subsection{Jet flavor tagging selection}

\begin{table}[ht]
  \caption{\btag selection criteria.}%
  \label{tab:object:btag}
  \centering
  % \resizebox{\textwidth}{!}{
  \begin{tabular}{ll}
    \toprule
    Feature & Criterion \\ 
    \midrule
    & EM Topo Jets / Track jets / VR jets \\
    \midrule
    Jet collection  & \texttt{AntiKt4EMTopo/AntiKt2PV0/AntiKtVR30Rmax4Rmin02} \\
    Jet selection   & \(\pT > \qty[parse-numbers=false]{XX}{\GeV}\) \\
    	            	& \(|\eta| < X\) \\				
                    & JVT cut if applicable \\
    \midrule
    Algorithm 		  & \texttt{MV2c10/MV2c10mu/MV2c10rnn/DL1/DL1mu/DL1rnn} \\
    \midrule
    Operating point & Hybrid /  Fixed \\
                    & Eff = 60 / 70 / 77 / 85 \\
    CDI             & \texttt{2017-21-13TeV-MC16-CDI-2017-12-22\_v1} \\
  \bottomrule
  \end{tabular}
  % }
\end{table}%

\subsection{Track selection}

If you use tracks as particular objects on which you cut in your analysis.

\begin{table}[ht]
  \caption{\texttt{TrackParticle} object selection criteria.}%
  \label{tab:object:track}
  \centering
  % \resizebox{\textwidth}{!}{
  \begin{tabular}{ll}
    \toprule
    Tracking algorithm								    & Primary / Large Radius Tracking / Custom \\
    Track quality selection (official)    & \texttt{Loose/Tight} \\
    \pT                                   & \(> \qty[parse-numbers=false]{XX}{\GeV}\) \\
    \(|\eta|\)                            & \(< X\) \\
    Track-vertex association criteria     & \texttt{Loose/Tight} \\
    Track-to-tet association method       & Ghost Matched / \(\Delta R\) \\
    \bottomrule
  \end{tabular}
  % }
\end{table}

\subsection{Overlap removal}

The reconstruction of the same energy deposits as multiple objects is resolved using the standard overlap removal tools, \texttt{AssociationUtils}, documented \href{https://gitlab.cern.ch/atlas/athena/blob/21.2/PhysicsAnalysis/AnalysisCommon/AssociationUtils/README.rst}{here}

The (Standard/Heavy-flavor/Boosted/Boosted+Heavy-flavor/lepton-favored) working point is used corresponding to:

\begin{table}[ht]
  % \resizebox{\textwidth}{!}{
  \begin{tabular}{lll}
    \toprule
    Reject & Against & Criteria \\
    \midrule
    Electron & Electron & shared track, \(\pTX[][1] < \pTX[][2]\) \\
    Tau      & Electron & \(\Delta R <\) 0.2 \\
    Tau      & Muon     & \(\Delta R <\) 0.2 \\
    Muon     & Electron & is Calo-Muon and shared ID track \\
    Electron & Muon     & shared ID track \\
    Photon   & Electron & \(\Delta R < 0.4\) \\
    Photon   & Muon     & \(\Delta R < 0.4\) \\
    Jet      & Electron & [\(\Delta R < 0.2\) / Not a \bjet and \(\Delta R <\) 0.2] \\
    Electron & Jet      & [\(\Delta R < 0.4\) / \(\Delta R < \min(0.4, 0.04 + \qty{10}{\GeV}/\pT(e))\)/None] \\
    Jet      & Muon     & [\(\texttt{NumTrack} < 3\) and (ghost-associated or \(\Delta R < 0.2\)) / \\
                       && not a \bjet and \(\texttt{NumTrack} < 3\) and (ghost-associated or \(\Delta R < 0.2\))] \\
    Muon     & Jet      & [\(\Delta R < 0.4\) / \(\Delta R < \min(0.4, 0.04 + \qty{10}{\GeV}/\pT(\mu))\)/None] \\
    Jet      & Tau      & \(\Delta R < 0.2\) \\
    Photon   & Jet      & \(\Delta R < 0.4\) \\
    Fat-jet  & Electron & \(\Delta R < 1.0\) \\
    Jet      & Fat-jet  & \(\Delta R < 1.0\) \\
    \bottomrule
  \end{tabular}
  % }
\end{table}

\(\Delta R\) is calculated using rapidity by default.





\section{Event selection}
\textit{The following items should also be filled in for the event selection.  There may, of course, 
still be some minor open items, as long as they don’t significantly affect the analysis strategy, 
but these should be well defined and clearly indicated (e.g. coloured/bold) in the text in this 
section and in the list of outstanding tasks within the executive summary. 
Both should be updated as the analysis progresses.}

\subsection{Event cleaning}
Following the \href{https://twiki.cern.ch/twiki/bin/viewauth/Atlas/DataPreparationCheckListForPhysicsAnalysis}{recommendations of the DataPrep group}, the following event-level requirements are made.

We use the official GRL\@:
 \begin{verbatim} FILL IN HERE \end{verbatim}
 
The following event-level vetos are made to reject bad / corrupt events:
 \begin{itemize}
  \item LAr noise burst and data corruption (\verb|xAOD::EventInfo::LAr|),
  \item Tile corrupted events (\verb|xAOD::EventInfo::Tile|),
  \item events affected by the SCT recovery procedure for single event upsets (\verb|xAOD::EventInfo::SCT|),
  \item incomplete events (\verb|xAOD::EventInfo::Core|).
 \end{itemize}
 
 Debug stream events [have/have not] been included.
 
 Checks [have/have not] been done to remove duplicate events.
 
Events are required to have a primary vertex with at least two associated tracks.
The primary vertex is selected as the one with the largest \(\Sigma \pT^2\),
where the sum is over all tracks with transverse momentum \(\pT > \qty{0.4}{\GeV}\) that are associated with the vertex.
 
 


\section{Background Modelling}
\textit{After outlining the object and event selection, noting possible outstanding points that still
need to be addressed to freeze the selection, you should demonstrate that you can analyze
the dataset that you intend to publish.  This should include CR/VR plots for
the main backgrounds with the full data (full run-2 analyses) or at least a representative
majority of the data (analyses during data-taking); for the more minor backgrounds this may
still be in progress but an outline of the planned method should be present.}


\section{Systematic Uncertainties}

\textit{Several systematics may still be missing but the note should include a proposed plan listing the CP systematics 
you will need to consider in this analysis (+ timescale on which they will be available if not already) and 
an outline of how the systematics on the backgrounds are proposed to be determined. If not statistics-limited, 
the most dominant systematic(s) should be present.}

\include{systematics}

\section{Statistical Model/Results}

\textit{An overview of the final fit setup including the final discriminating variables(s),
the (SR/CR) regions to be included in the fit and the floating normalisation parameters.
Some rough first  expected limits/discovery sensitivity plots are useful if you have them but
not necessary. In this case the binning of the final variable(s) and the systematics
smoothing/pruning should be indicated.}


\end{document}
