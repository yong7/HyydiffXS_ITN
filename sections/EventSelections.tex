\section{Object Definitions and Event Selections}
\label{sec:EventSelections}

In this section, we provide a summary of the different object definitions and event selections. %More details are given in the common performance note~\cite{Nomidis:2718255} and in the STXS/coupling note~\cite{Berger:2764716}.

\subsection{Object Defintions}
\label{sec:objectdefinitions}

\subsubsection{Photon Reconstruction and pre-selections}
\label{sssec:photon_preselection}
Photons are reconstructed using dynamic, variable-size, topological energy clusters in the electromagnetic calorimeter~\cite{ATL-PHYS-PUB-2017-022}. Converted photons are reconstructed from clusters associated to inner detector track(s) matched to a conversion vertex. Photon candidates with associated super-clusters are deemed to be of bad quality and excluded from the analysis if: 1. they are affected by dead front-end boards in the first or second sampling layer, 2. there are high-voltage trips present in any of the three sampling layers, or 3. they include masked cells in the core. In addition, a photon candidate is not considered if any of the eight central strips is masked.
Loose photons are defined from the Loose identification criteria, which are based on shower-shape-variables, defined using cells in the middle and final layer of the LAr accordion calorimeter. We pre-select loose photons requiring $\pT>\SI{25}{\GeV}$ and $|\eta_{S2}|<2.37$\footnote{$\eta_{S2}$ is the $\eta$ position of the calorimetric cluster in the second sampling of the EM calorimeter.} and vetoing the transition region between the barrel and endcap sections at $1.37 < |\eta_{S2}| < 1.52$. %The \pT used for the kinematic cuts and the invariant mass computation takes into account the correction of the $\eta$ from the diphoton primary vertex position.
The four-momenta used for the computation of the invariant mass, as well as kinematic cuts, are corrected for the position of the diphoton primary vertex.

Photons are calibrated using the latest Run-3 calibration corrections for the energy scale and resolution detailed in~\cite{EGAM-2021-02}. These corrections are tagged with the model~\texttt{es2024\_Run3\_v0} and correspond to \texttt{TUNE23}. Unless otherwise specified, the results reported in this document are obtained with this model. \footnote{\textcolor{blue}{The latest calibration scheme is used, the one employed in h033, which is likely the tag to be used for Moriond.}} 
The~\texttt{1NP\_v1} systematics scheme is used which contains two energy scale related uncertainties and one resolution uncertainty.
%which contains 70 variations for the scale and 9 for the resolution is used in this analysis.

In order to reduce the background coming from neutral mesons decaying to photon pairs and charged particles from pileup, an isolation requirement is applied using separate calorimeter-based and track-based requirements. The \texttt{FixedCutLoose} working point is used. The requirements of this working point are specified in \Sect{\ref{sec:eventselections}}.

During the event selection, events are required to have at least two loose photons, and the two loose photons with the highest \pT define the Higgs candidate.

\paragraph{Diphoton vertex}
A Neural Network (NN) is used to identify the correct diphoton vertex among the primary vertices reconstructed in ATLAS. The vertex with the highest score for this NN is chosen to be the primary vertex in this analysis. The NN combines information from the longitudinal segmentation of the calorimeter (`pointing') for the two photons, the $\Delta\phi$ between the tracks at the vertices under consideration and the diphoton system, as well as the scalar sums of track-\pT and \pTsq. Two NNs are trained, one for unconverted photon pairs and one for pairs with at least one converted photon. The NNs perform better than the usual hardest vertex requirement at identifying the truth vertex within \SI{0.3}{\milli\meter}. They are particularly performant for processes with few tracks in the final state, such as \ggH production. 
%A comparison study of the vertex selection efficiencies for various production modes was performed in the context of the \Hyy STXS/couplings analysis~\cite{Berger:2764716}: it has been shown that the efficiency of the \ggH production modes increases from 54\% to 65\% when the NN vertex is used in place of the usual hardest one as shown in \Fig{\ref{fig:myy_vertex_resolution}}.

Once the diphoton vertex has been identified, all the object kinematics are recomputed and all the selections reported in the following sections are relative to this diphoton vertex. The identification of the primary vertex pointed to by the two photons improves the inclusive \myy resolution by 8\%.
% , as can be observed in \Fig{\ref{fig:myy_vertex_resolution}}.
The improvement is driven by the dominant \ggH production mode, where the difference in selection efficiency between the standard hardest vertex requirement and the NN classification is the largest.
%\begin{figure}[tbp]
%	\centering
%    \includegraphics[width=0.49\linewidth]{selection/PV_mode}    
%	\includegraphics[width=0.49\linewidth]{selection/vtx_resolution}
%	\caption{(Left) NN vertex identification efficiencies compared to hardest vertex ones for the different Higgs production modes. The criteria of identification is $|z_{truth} - z_{vtx}| < \SI{0.3}{\milli\meter}$. (Right) Diphoton invariant mass distributions for the events that pass the full analysis selection described in \Sect{\ref{ssec:event_selection}} when the selected primary vertex is the nominal ATLAS vertex (``Hardest") compared to the diphoton Neural Network one.}
%	\label{fig:myy_vertex_resolution}
%\end{figure}


\subsubsection{Electron, muon, jet and \MET selections}
The reconstruction and selection of these objects is common to other SM \Hyy analyses and is described in great detail in the STXS/coupling note \cite{Berger:2764716} from Run 2. Updated calibration and identification/tagging maps are specified below, as well as a reminder of the baseline object selection cuts.  

\subsubsection{Overlap removal between objects}
In order to avoid any possible double counting between objects, overlap removal is applied following a common \Hyy strategy, starting with the selected photons. The specific recommendations for analysis harmonization are detailed in~\cite{Adams:1700874}. Following this approach, the two leading photons are always kept. Electrons and muons within a $\Delta R=0.4$ cone around any of the photons are discarded. Jets with $\Delta R<0.2$ ($\Delta R<0.4$) to an electron (photon) are excluded, as are electrons with $\Delta R<0.4$ to the remaining jets. Lastly, muons with $\Delta R<0.4$ to a jet are removed. 

\subsubsection{Summary of Combined Performance recommendations used by the analysis}
In this section a summary of the recommendations from the Combined Performance groups relevant to the analysis is provided. 
%These exactly match the recommendation used in the recent STXS/coupling analysis \cite{Berger:2764716} a part for the photons which have the updated final EGamma scale calibrations. \\


\textbf{Photons}
\begin{itemize}
	\item ESModel: \path{es2024_Run3_v0}
	\item Shower shape fudge factors: \path{ElectronPhotonShowerShapeFudgeTool/}\\ \path{v2/PhotonFudgeFactors.root}
	\item ID: Tight (for signal region);  \path{ElectronPhotonSelectorTools/offline/20180825/}\\ \path{PhotonIsEMTightSelectorCutDefs.conf}
	\item ID scale factors: \path{PhotonEfficiencyCorrection/2015_2025/rel22.2/2022_Summer_Prerecom_v1/map0.txt}
	\item Isolation working point: \path{FixedCutLoose}
	\item Ambiguity Tool: \path{AuthorPhoton} and \path{AuthorAmbiguous} retained
	\item Photon cleaning: \path{DFCommonPhotonsCleaning}
	\item Photon quality: removed \path{BADCLUSPHOTON} and \path{deadHVTool}    
\end{itemize}

\textbf{Electrons}
\begin{itemize}
	\item Selection: $p_{T} > 10 \GeV, |\eta|<2.47, 1.37 < |\eta| < 1.52, |d_{0}/\sigma(d_{0})|<5, |z_{0}sin\theta|<0.5$ mm, MediumLH 
	\item ESModel: \path{es2024_Run3_v0}
	\item ID scale factors: \path{ElectronEfficiencyCorrection/2015_2025/rel22.2/2025_Run3_Consolidated_Prerecom_v3/offline/efficiencySF.offline.MediumLLH_d0z0_v18.root}
	\item Isolation working point: \path{LooseVarRad}
	\item Electron quality: removed \path{BADCLUSELECTRON} and \path{deadHVTool}    
\end{itemize}

\textbf{Muons}
\begin{itemize}
	\item Selection: $p_{T} > 5 \GeV, |\eta|<2.7, |d_{0}/\sigma(d_{0})|<3, |z_{0}sin\theta|<0.5$ mm, Medium ID
  \item Calibration release: \path{Recs2025_03_26_Run2Run3}
	\item Efficiency release: \path{250418_Preliminary_r24run3}
	\item Isolation working point: \path{PflowLoose_VarRad}
\end{itemize}
\textbf{Jets}
\begin{itemize}
	\item Baseline selection: $p_{T} > 20 \GeV, |y|<4.4$, 
  \item ID: 
  \begin{itemize}
    \item JVT, FixedEffPt \path{JetJvtEfficiency/May2024/NNJvtSFFile_Run3_EMPFlow.root}; 
    \item fJVT: Loose \path{JetJvtEfficiency/May2024/fJvtSFFile_Run2_EMPFlow.root} 
  \end{itemize}
  \item Calibration file: \path{AntiKt4EMPFlow_MC23a_PreRecR22_Phase2_CalibConfig_ResPU_EtaJES_GSC_241208_InSitu.config}
  \item Calibration sequence: \path{JetArea_Residual_EtaJES_GSC}
  \item Calibration area: \path{00-04-83}
  \item Uncertainty scheme: \path{rel22/Winter2025_PreRec/R4_CategoryReduction_FullJER.config}
\end{itemize}

\textbf{B-jets}

\begin{itemize}
	\item Baseline selection: $p_{T} > 20 \GeV, |\eta|<2.5$, 
  \item CDI file: \path{13p6TeV/MC23_2025-06-17_GN2v01_v4.root}
  \item Tagger name: \texttt{GN2v01}
  \item Tagger WP: \texttt{GN2v01\_FixedCutBEff\_85} 
 
\end{itemize}

\subsubsection{Corrections to MC samples}
\label{sssec:corrections_MC}

Corrections are applied to the MC samples in order to improve agreement with the data:

\begin{itemize}

\item Shifts are applied to the photon shower shape variables and to the photon calorimeter isolation~\cite{PERF-2017-02,EGAM-2018-01}. Such shifts are determined by the EGamma group as the average data-MC difference in photon-enriched control samples.

\item Shifts are applied to the isolation energy of photons (called `data-driven shifts') in order to correct a consistent discrepancy between data and MC on the peak position of the isolation energy distribution. This is applied before computing the scale factors.

\item Identification and isolation efficiency scale factors for photons~\cite{PERF-2017-02,EGAM-2018-01}. Energy resolution corrections for all simulated photons are also taken into account~\cite{EGAM-2018-01,PERF-2017-03}, while energy scale corrections are applied to data.
\end{itemize}




% - define the various physics objects reconstructed in the analysis in line withe the latest CP recommendations


\subsection{Event and Object Selections}
\label{sec:eventselections}
% - define the basic event and object selection 
% - define the fiducial sub-region
% -- define truth-level objects used in MC

The full event selection consists of a few extra requirements on the
two leading-\pT\ photons after the preselection applied during the MxAOD
production is performed. With the exception of the photon identification requirements, the final selection at the detector level resembles closely the fiducial selection applied at the particle level.

We summarise here the full event selection, applied following the preselection described in Section~\ref{sec:samples}. We remind the reader that the energy and  momentum scales of all particles (photons, leptons, jets) have been calibrated. In  simulation, smearing is applied to match the energy and momentum resolution in the data Scale factors, determined from auxiliary measurements, are applied in simulation to account for data-MC differences in ID/tagging efficiency for the various particle hypotheses.

The event selection is the following:
\begin{itemize}
\item The two photon candidates with the highest transverse momentum are matched to the photon trigger objects of the trigger used in the event (the unprescaled diphoton trigger is a 2-part trigger; it has a minimum L1 level energy threshold of 20 GeV, and HLT-level asymmetric \pT\ thresholds of 35 and 25~GeV). These two photon candidates are used to reconstruct the Higgs boson candidate

\item The two selected photons should pass tight identification requirements. Photons that are located at an absolute pseudorapidity greater than 2.37 or in the transition region between the barrel and endcap sections of the LAr calorimeter are discarded. In the former region ($1.37<|\eta|<1.52$), the transverse segmentation of the first layer of the accordion calorimeter is too coarse to allow prompt photons to be distinguished from neutral pions.

\item The two selected photons are required to pass the `FixedCutLoose' isolation requirements: the total transverse isolation energy in the calorimeter $\ET^\mathrm{iso}$ in a cone of radius 0.2 around the photon direction must be less than 6.5\% of the photon \pT, likewise the scalar sum of \pT\ of the tracks from the diphoton primary vertex in a cone of radius 0.2 around the photon candidate must be less than 5\% of the photon \pT.

\item The leading and sub-leading photons are required to have $\pT/m_{\gamma\gamma}$ larger than 0.35 and 0.25, respectively. This long-standing selection is revised in order to provide a more theory-friendly approach in Appendix~\ref{app:productcut}.

\item The diphoton invariant mass must be in the range $105 < m_{\gamma\gamma} < 160$~GeV.
\end{itemize}

The selection is based on Run 3 pre-recommendations.
In particular, pre-recommendations for photons (including efficiency, isolation and trigger scale factors, as well as calibrations) are considered.

The number of events selected in the full Run3 data is~1532642. The cutflow for events in data is given in Table ~\ref{tab:cutflow_data}.



\begin{table}[!htbp]
\caption{Selection cutflow in data}
\label{tab:cutflow_data}
\begin{tabular}{lrr  rr  rr}
\toprule
Selection step & \multicolumn{2}{c}{2022} & \multicolumn{2}{c}{2023} & \multicolumn{2}{c}{2024} \\
 & Events & rel.eff.(\%) & Events & rel.eff.(\%) & Events & rel.eff.(\%) \\
\midrule
$N_{\mathrm{xAOD}}$ & 3311547552 & 100.00 & 2641913539 & 100.00 & 13425206475 & 100.00 \\
$N_{\mathrm{DxAOD}}$ & 163127048 & 4.93 & 88717130 & 3.36 & 257658730 & 1.92 \\
All events & 163127048 & 100.00 & 88717130 & 100.00 & 257658730 & 100.00 \\
No duplicates & 163127048 & 100.00 & 88717128 & 100.00 & 257658711 & 100.00 \\
GRL & 134189883 & 82.26 & 81895650 & 92.31 & 249864992 & 96.98 \\
Pass trigger & 124604672 & 92.86 & 61747845 & 75.40 & 214472114 & 85.84 \\
Detector DQ & 124597123 & 99.99 & 61743925 & 99.99 & 214418731 & 99.98 \\
Has PV & 124597110 & 100.00 & 61743925 & 100.00 & 214418729 & 100.00 \\
2 loose photons & 13994099 & 11.23 & 14966482 & 24.24 & 56619654 & 26.41 \\
$e-\gamma$ ambiguity & 13994099 & 100.00 & 14966482 & 100.00 & 56617705 & 100.00 \\
Trigger match & 9214126 & 65.84 & 8386969 & 56.04 & 32246113 & 56.95 \\
tight ID & 3087071 & 33.50 & 2903849 & 34.62 & 11102057 & 34.43 \\
isolation & 1382065 & 44.77 & 1293714 & 44.55 & 5001065 & 45.05 \\
rel. $p_{T}$ cuts & 1195649 & 86.51 & 1115754 & 86.24 & 4316498 & 86.31 \\
$m_{\gamma\gamma} \in [105,160]$~GeV & 280376 & 23.45 & 256835 & 23.02 & 995431 & 23.06 \\
\midrule
Efficiency(\%) & \multicolumn{2}{c}{0.0085} & \multicolumn{2}{c}{0.0097} & \multicolumn{2}{c}{0.0074} \\
\bottomrule
\end{tabular}
\end{table}


The cutflows for the various signal are shown in Table ~\ref{tab:cutflow_ggH}--\ref{tab:cutflow_bbH}. . The cutflow for the diphoton background is given in Table~\ref{tab:cutflow_yyjets}.
\begin{table}[!htbp]
\caption{Selection cutflow in ggH Signal MC. Events are weighted}
\label{tab:cutflow_ggH}
\begin{tabular}{lrr  rr  rr}
\toprule
Selection step & \multicolumn{2}{c}{mc23a} & \multicolumn{2}{c}{mc23d} & \multicolumn{2}{c}{mc23e} \\
 & Yields & rel.eff.(\%) & Yields & rel.eff.(\%) & Yields & rel.eff.(\%) \\
\midrule
$N_{\mathrm{xAOD}}$ & 2259.40 & 100.00 & 1960.36 & 100.00 & 7869.05 & 100.00 \\
$N_{\mathrm{DxAOD}}$ & 2259.40 & 100.00 & 1960.36 & 100.00 & 7869.05 & 100.00 \\
All events & 3718.40 & 164.57 & 3225.91 & 164.56 & 12952.95 & 164.61 \\
No duplicates & 3718.40 & 100.00 & 3225.91 & 100.00 & 12952.95 & 100.00 \\
GRL & 3718.40 & 100.00 & 3225.91 & 100.00 & 12952.95 & 100.00 \\
Pass trigger & 2490.24 & 66.97 & 2151.67 & 66.70 & 8497.69 & 65.60 \\
Detector DQ & 2490.24 & 100.00 & 2151.67 & 100.00 & 8497.69 & 100.00 \\
Has PV & 2490.24 & 100.00 & 2151.67 & 100.00 & 8497.69 & 100.00 \\
2 loose photons & 2052.09 & 82.41 & 1765.29 & 82.04 & 7019.73 & 82.61 \\
$e-\gamma$ ambiguity & 2052.09 & 100.00 & 1765.29 & 100.00 & 7000.15 & 99.72 \\
Trigger match & 1837.31 & 89.53 & 1586.34 & 89.86 & 6297.08 & 89.96 \\
tight ID & 1604.09 & 87.31 & 1367.16 & 86.18 & 5414.44 & 85.98 \\
isolation & 1465.35 & 91.35 & 1235.69 & 90.38 & 4863.20 & 89.82 \\
rel. $p_{T}$ cuts & 1366.22 & 93.23 & 1152.22 & 93.25 & 4536.81 & 93.29 \\
$m_{\gamma\gamma} \in [105,160]$~GeV & 1365.82 & 99.97 & 1151.90 & 99.97 & 4535.32 & 99.97 \\
\midrule
Efficiency(\%) & \multicolumn{2}{c}{60.45} & \multicolumn{2}{c}{58.76} & \multicolumn{2}{c}{57.63} \\
\bottomrule
\end{tabular}
\end{table}


\begin{table}[!htbp]
\caption{Selection cutflow in VBF Signal MC. Events are weighted}
\label{tab:cutflow_VBF}
\begin{tabular}{lrr  rr  rr}
\toprule
Selection step & \multicolumn{2}{c}{mc23a} & \multicolumn{2}{c}{mc23d} & \multicolumn{2}{c}{mc23e} \\
 & Yields & rel.eff.(\%) & Yields & rel.eff.(\%) & Yields & rel.eff.(\%) \\
\midrule
$N_{\mathrm{xAOD}}$ & 290.71 & 100.00 & 252.31 & 100.00 & 1012.50 & 100.00 \\
$N_{\mathrm{DxAOD}}$ & 290.71 & 100.00 & 252.31 & 100.00 & 1012.50 & 100.00 \\
All events & 290.44 & 99.91 & 251.98 & 99.87 & 1011.76 & 99.93 \\
No duplicates & 290.44 & 100.00 & 251.98 & 100.00 & 1011.76 & 100.00 \\
GRL & 290.44 & 100.00 & 251.98 & 100.00 & 1011.76 & 100.00 \\
Pass trigger & 208.15 & 71.67 & 180.00 & 71.43 & 714.18 & 70.59 \\
Detector DQ & 208.15 & 100.00 & 180.00 & 100.00 & 714.18 & 100.00 \\
Has PV & 208.15 & 100.00 & 180.00 & 100.00 & 714.18 & 100.00 \\
2 loose photons & 162.67 & 78.15 & 140.09 & 77.83 & 558.98 & 78.27 \\
$e-\gamma$ ambiguity & 162.67 & 100.00 & 140.09 & 100.00 & 557.57 & 99.75 \\
Trigger match & 146.16 & 89.85 & 126.40 & 90.22 & 503.47 & 90.30 \\
tight ID & 127.31 & 87.10 & 108.86 & 86.12 & 432.58 & 85.92 \\
isolation & 117.51 & 92.30 & 99.66 & 91.55 & 394.26 & 91.14 \\
rel. $p_{T}$ cuts & 106.37 & 90.52 & 90.23 & 90.54 & 356.86 & 90.52 \\
$m_{\gamma\gamma} \in [105,160]$~GeV & 106.23 & 99.87 & 90.11 & 99.87 & 356.39 & 99.87 \\
\midrule
Efficiency(\%) & \multicolumn{2}{c}{36.54} & \multicolumn{2}{c}{35.71} & \multicolumn{2}{c}{35.20} \\
\bottomrule
\end{tabular}
\end{table}


\begin{table}[!htbp]
\caption{Selection cutflow in $W^+H$ Signal MC. Events are weighted}
\label{tab:cutflow_WpH}
\begin{tabular}{lrr  rr  rr}
\toprule
Selection step & \multicolumn{2}{c}{mc23a} & \multicolumn{2}{c}{mc23d} & \multicolumn{2}{c}{mc23e} \\
 & Yields & rel.eff.(\%) & Yields & rel.eff.(\%) & Yields & rel.eff.(\%) \\
\midrule
$N_{\mathrm{xAOD}}$ & 63.55 & 100.00 & 54.91 & 100.00 & 220.55 & 100.00 \\
$N_{\mathrm{DxAOD}}$ & 63.55 & 100.00 & 54.91 & 100.00 & 220.55 & 100.00 \\
All events & 63.22 & 99.48 & 54.85 & 99.89 & 220.23 & 99.85 \\
No duplicates & 63.22 & 100.00 & 54.85 & 100.00 & 220.23 & 100.00 \\
GRL & 63.22 & 100.00 & 54.85 & 100.00 & 220.23 & 100.00 \\
Pass trigger & 40.58 & 64.19 & 35.21 & 64.20 & 139.57 & 63.38 \\
Detector DQ & 40.58 & 100.00 & 35.21 & 100.00 & 139.57 & 100.00 \\
Has PV & 40.58 & 100.00 & 35.21 & 100.00 & 139.57 & 100.00 \\
2 loose photons & 29.82 & 73.48 & 25.77 & 73.17 & 102.86 & 73.70 \\
$e-\gamma$ ambiguity & 29.82 & 100.00 & 25.77 & 100.00 & 102.53 & 99.68 \\
Trigger match & 26.80 & 89.86 & 23.28 & 90.36 & 92.74 & 90.45 \\
tight ID & 23.17 & 86.45 & 19.88 & 85.39 & 79.16 & 85.36 \\
isolation & 20.82 & 89.89 & 17.71 & 89.07 & 70.37 & 88.89 \\
rel. $p_{T}$ cuts & 18.99 & 91.17 & 16.15 & 91.21 & 64.18 & 91.20 \\
$m_{\gamma\gamma} \in [105,160]$~GeV & 18.86 & 99.34 & 16.04 & 99.31 & 63.77 & 99.37 \\
\midrule
Efficiency(\%) & \multicolumn{2}{c}{29.68} & \multicolumn{2}{c}{29.21} & \multicolumn{2}{c}{28.91}\\
\bottomrule
\end{tabular}
\end{table}


\begin{table}[!htbp]
\caption{Selection cutflow in $W^-H$ Signal MC. Events are weighted}
\label{tab:cutflow_WmH}
\begin{tabular}{lrr  rr  rr}
\toprule
Selection step & \multicolumn{2}{c}{mc23a} & \multicolumn{2}{c}{mc23d} & \multicolumn{2}{c}{mc23e} \\
 & Yields & rel.eff.(\%) & Yields & rel.eff.(\%) & Yields & rel.eff.(\%) \\
\midrule
$N_{\mathrm{xAOD}}$ & 40.42 & 100.00 & 35.10 & 100.00 & 140.74 & 100.00 \\
$N_{\mathrm{DxAOD}}$ & 40.42 & 100.00 & 35.10 & 100.00 & 140.74 & 100.00 \\
All events & 40.37 & 99.88 & 35.02 & 99.78 & 140.63 & 99.92 \\
No duplicates & 40.37 & 100.00 & 35.02 & 100.00 & 140.63 & 100.00 \\
GRL & 40.37 & 100.00 & 35.02 & 100.00 & 140.63 & 100.00 \\
Pass trigger & 28.09 & 69.59 & 24.34 & 69.50 & 96.20 & 68.41 \\
Detector DQ & 28.09 & 100.00 & 24.34 & 100.00 & 96.20 & 100.00 \\
Has PV & 28.09 & 100.00 & 24.34 & 100.00 & 96.20 & 100.00 \\
2 loose photons & 21.36 & 76.04 & 18.42 & 75.67 & 73.25 & 76.14 \\
$e-\gamma$ ambiguity & 21.36 & 100.00 & 18.42 & 100.00 & 73.03 & 99.71 \\
Trigger match & 19.16 & 89.68 & 16.59 & 90.05 & 65.86 & 90.18 \\
tight ID & 16.60 & 86.63 & 14.19 & 85.52 & 56.27 & 85.43 \\
isolation & 14.92 & 89.89 & 12.66 & 89.24 & 50.04 & 88.93 \\
rel. $p_{T}$ cuts & 13.61 & 91.25 & 11.53 & 91.09 & 45.68 & 91.29 \\
$m_{\gamma\gamma} \in [105,160]$~GeV & 13.54 & 99.44 & 11.47 & 99.42 & 45.40 & 99.39 \\
\midrule
Efficiency(\%) & \multicolumn{2}{c}{33.49} & \multicolumn{2}{c}{32.66} & \multicolumn{2}{c}{32.26} \\
\bottomrule
\end{tabular}
\end{table}


\begin{table}[!htbp]
\caption{Selection cutflow in $q\bar{q}\to ZH$ Signal MC. Events are weighted}
\label{tab:cutflow_ZH}
\begin{tabular}{lrr  rr  rr}
\toprule
Selection step & \multicolumn{2}{c}{mc23a} & \multicolumn{2}{c}{mc23d} & \multicolumn{2}{c}{mc23e} \\
 & Yields & rel.eff.(\%) & Yields & rel.eff.(\%) & Yields & rel.eff.(\%) \\
\midrule
$N_{\mathrm{xAOD}}$ & 57.55 & 100.00 & 50.04 & 100.00 & 200.64 & 100.00 \\
$N_{\mathrm{DxAOD}}$ & 57.55 & 100.00 & 50.04 & 100.00 & 200.64 & 100.00 \\
All events & 57.47 & 99.86 & 49.86 & 99.63 & 200.19 & 99.78 \\
No duplicates & 57.47 & 100.00 & 49.86 & 100.00 & 200.19 & 100.00 \\
GRL & 57.47 & 100.00 & 49.86 & 100.00 & 200.19 & 100.00 \\
Pass trigger & 38.41 & 66.84 & 33.14 & 66.46 & 131.26 & 65.57 \\
Detector DQ & 38.41 & 100.00 & 33.14 & 100.00 & 131.26 & 100.00 \\
Has PV & 38.41 & 100.00 & 33.14 & 100.00 & 131.26 & 100.00 \\
2 loose photons & 29.06 & 75.67 & 25.01 & 75.48 & 99.75 & 75.99 \\
$e-\gamma$ ambiguity & 29.06 & 100.00 & 25.01 & 100.00 & 99.45 & 99.70 \\
Trigger match & 26.08 & 89.74 & 22.56 & 90.22 & 89.83 & 90.33 \\
tight ID & 22.58 & 86.57 & 19.30 & 85.52 & 76.67 & 85.35 \\
isolation & 20.32 & 89.98 & 17.21 & 89.17 & 68.21 & 88.96 \\
rel. $p_{T}$ cuts & 18.51 & 91.10 & 15.67 & 91.06 & 62.17 & 91.14 \\
$m_{\gamma\gamma} \in [105,160]$~GeV & 18.42 & 99.53 & 15.59 & 99.51 & 61.87 & 99.52 \\
\midrule
Efficiency(\%) & \multicolumn{2}{c}{32.01} & \multicolumn{2}{c}{31.16} & \multicolumn{2}{c}{30.84}\\
\bottomrule
\end{tabular}
\end{table}

\begin{table}[!htbp]
\caption{Selection cutflow in $gg\to ZH$ Signal MC. Events are weighted}
\label{tab:cutflow_ggZH}
\begin{tabular}{lrr  rr  rr}
\toprule
Selection step & \multicolumn{2}{c}{mc23a} & \multicolumn{2}{c}{mc23d} & \multicolumn{2}{c}{mc23e} \\
 & Yields & rel.eff.(\%) & Yields & rel.eff.(\%) & Yields & rel.eff.(\%) \\
\midrule
$N_{\mathrm{xAOD}}$ & 9.69 & 100.00 & 8.40 & 100.00 & 33.73 & 100.00 \\
$N_{\mathrm{DxAOD}}$ & 9.69 & 100.00 & 8.40 & 100.00 & 33.73 & 100.00 \\
All events & 9.69 & 99.99 & 8.40 & 99.99 & 33.74 & 100.02 \\
No duplicates & 9.69 & 100.00 & 8.40 & 100.00 & 33.74 & 100.00 \\
GRL & 9.69 & 100.00 & 8.40 & 100.00 & 33.74 & 100.00 \\
Pass trigger & 8.13 & 83.98 & 7.03 & 83.62 & 28.06 & 83.15 \\
Detector DQ & 8.13 & 100.00 & 7.03 & 100.00 & 28.06 & 100.00 \\
Has PV & 8.13 & 100.00 & 7.03 & 100.00 & 28.06 & 100.00 \\
2 loose photons & 6.08 & 74.69 & 5.22 & 74.24 & 20.83 & 74.25 \\
$e-\gamma$ ambiguity & 6.08 & 100.00 & 5.22 & 100.00 & 20.78 & 99.78 \\
Trigger match & 5.57 & 91.60 & 4.80 & 91.94 & 19.09 & 91.85 \\
tight ID & 4.83 & 86.71 & 4.12 & 85.88 & 16.30 & 85.38 \\
isolation & 4.41 & 91.45 & 3.75 & 90.95 & 14.81 & 90.88 \\
rel. $p_{T}$ cuts & 3.96 & 89.65 & 3.35 & 89.44 & 13.32 & 89.90 \\
$m_{\gamma\gamma} \in [105,160]$~GeV & 3.94 & 99.47 & 3.33 & 99.44 & 13.24 & 99.44 \\
\midrule
Efficiency(\%) & \multicolumn{2}{c}{40.63} & \multicolumn{2}{c}{39.65} & \multicolumn{2}{c}{39.26} \\
\bottomrule
\end{tabular}
\end{table}


\begin{table}[!htbp]
\caption{Selection cutflow in $t\bar{t}H$ Signal MC. Events are weighted}
\label{tab:cutflow_ttH}
\begin{tabular}{lrr  rr  rr}
\toprule
Selection step & \multicolumn{2}{c}{mc23a} & \multicolumn{2}{c}{mc23d} & \multicolumn{2}{c}{mc23e} \\
 & Yields & rel.eff.(\%) & Yields & rel.eff.(\%) & Yields & rel.eff.(\%) \\
\midrule
$N_{\mathrm{xAOD}}$ & 40.63 & 100.00 & 35.16 & 100.00 & 141.25 & 100.00 \\
$N_{\mathrm{DxAOD}}$ & 40.63 & 100.00 & 35.16 & 100.00 & 141.25 & 100.00 \\
All events & 40.54 & 99.77 & 35.17 & 100.04 & 141.22 & 99.98 \\
No duplicates & 40.54 & 100.00 & 35.17 & 100.00 & 141.22 & 100.00 \\
GRL & 40.54 & 100.00 & 35.17 & 100.00 & 141.22 & 100.00 \\
Pass trigger & 34.88 & 86.05 & 30.12 & 85.65 & 119.55 & 84.65 \\
Detector DQ & 34.88 & 100.00 & 30.12 & 100.00 & 119.55 & 100.00 \\
Has PV & 34.88 & 100.00 & 30.12 & 100.00 & 119.55 & 100.00 \\
2 loose photons & 26.41 & 75.70 & 22.68 & 75.31 & 90.78 & 75.94 \\
$e-\gamma$ ambiguity & 26.41 & 100.00 & 22.68 & 100.00 & 90.61 & 99.82 \\
Trigger match & 23.64 & 89.52 & 20.45 & 90.17 & 81.55 & 90.00 \\
tight ID & 20.04 & 84.76 & 17.13 & 83.77 & 68.20 & 83.63 \\
isolation & 16.95 & 84.61 & 14.36 & 83.80 & 57.08 & 83.69 \\
rel. $p_{T}$ cuts & 15.34 & 90.47 & 12.99 & 90.50 & 51.69 & 90.55 \\
$m_{\gamma\gamma} \in [105,160]$~GeV & 15.16 & 98.87 & 12.84 & 98.83 & 51.07 & 98.81 \\
\midrule
Efficiency(\%) & \multicolumn{2}{c}{37.32} & \multicolumn{2}{c}{36.53} & \multicolumn{2}{c}{36.16} \\
\bottomrule
\end{tabular}
\end{table}


\begin{table}[!htbp]
\caption{Selection cutflow in $b\bar{b}H$ Signal MC. Events are weighted}
\label{tab:cutflow_bbH}
\begin{tabular}{lrr  rr  rr}
\toprule
Selection step & \multicolumn{2}{c}{mc23a} & \multicolumn{2}{c}{mc23d} & \multicolumn{2}{c}{mc23e} \\
 & Yields & rel.eff.(\%) & Yields & rel.eff.(\%) & Yields & rel.eff.(\%) \\
\midrule
$N_{\mathrm{xAOD}}$ & 37.47 & 100.00 & 32.50 & 100.00 & 130.62 & 100.00 \\
$N_{\mathrm{DxAOD}}$ & 37.47 & 100.00 & 32.50 & 100.00 & 130.62 & 100.00 \\
All events & 37.47 & 100.01 & 32.51 & 100.01 & 130.52 & 99.93 \\
No duplicates & 37.47 & 100.00 & 32.51 & 100.00 & 130.52 & 100.00 \\
GRL & 37.47 & 100.00 & 32.51 & 100.00 & 130.52 & 100.00 \\
Pass trigger & 26.56 & 70.90 & 22.49 & 69.20 & 89.62 & 68.66 \\
Detector DQ & 26.56 & 100.00 & 22.49 & 100.00 & 89.62 & 100.00 \\
Has PV & 26.56 & 100.00 & 22.49 & 100.00 & 89.62 & 100.00 \\
2 loose photons & 22.45 & 84.53 & 18.67 & 83.02 & 75.32 & 84.05 \\
$e-\gamma$ ambiguity & 22.45 & 100.00 & 18.67 & 100.00 & 75.12 & 99.74 \\
Trigger match & 20.07 & 89.38 & 16.68 & 89.32 & 67.55 & 89.91 \\
tight ID & 17.73 & 88.35 & 14.31 & 85.79 & 58.10 & 86.01 \\
isolation & 16.26 & 91.69 & 12.96 & 90.58 & 52.39 & 90.18 \\
rel. $p_{T}$ cuts & 15.30 & 94.13 & 12.14 & 93.68 & 49.25 & 94.02 \\
$m_{\gamma\gamma} \in [105,160]$~GeV & 15.30 & 99.97 & 12.14 & 99.97 & 49.24 & 99.97 \\
\midrule
Efficiency(\%) & \multicolumn{2}{c}{40.83} & \multicolumn{2}{c}{37.35} & \multicolumn{2}{c}{37.70} \\
\bottomrule
\end{tabular}
\end{table}


\begin{table}[!htbp]
\caption{Selection cutflow in $\gamma\gamma$+jets Background MC. Events are weighted}
\label{tab:cutflow_yyjets}
\begin{tabular}{lrr  rr  rr}
\toprule
Selection step & \multicolumn{2}{c}{mc23a} & \multicolumn{2}{c}{mc23d} & \multicolumn{2}{c}{mc23e} \\
 & Yields & rel.eff.(\%) & Yields & rel.eff.(\%) & Yields & rel.eff.(\%) \\
\midrule
$N_{\mathrm{xAOD}}$ & 1509524.64 & 100.00 & 1307366.97 & 100.00 & 5271766.50 & 100.00 \\
$N_{\mathrm{DxAOD}}$ & 1096259.43 & 72.62 & 939242.01 & 71.84 & 3504927.34 & 66.48 \\
All events & 1096616.73 & 100.03 & 941148.91 & 100.20 & 3497177.85 & 99.78 \\
No duplicates & 1096616.73 & 100.00 & 941148.91 & 100.00 & 3497177.85 & 100.00 \\
GRL & 1096616.73 & 100.00 & 941148.91 & 100.00 & 3497177.85 & 100.00 \\
Pass trigger & 1047880.91 & 95.56 & 895758.15 & 95.18 & 3482339.05 & 99.58 \\
Detector DQ & 1047880.91 & 100.00 & 895758.15 & 100.00 & 3482339.05 & 100.00 \\
Has PV & 1047880.91 & 100.00 & 895758.15 & 100.00 & 3482339.05 & 100.00 \\
2 loose photons & 807943.92 & 77.10 & 688376.01 & 76.85 & 2706833.26 & 77.73 \\
$e-\gamma$ ambiguity & 807943.92 & 100.00 & 688376.01 & 100.00 & 2697451.67 & 99.65 \\
Trigger match & 628563.10 & 77.80 & 534270.66 & 77.61 & 2105108.34 & 78.04 \\
tight ID & 551271.59 & 87.70 & 464333.38 & 86.91 & 1830583.38 & 86.96 \\
isolation & 452763.60 & 82.13 & 379321.00 & 81.69 & 1489304.13 & 81.36 \\
rel. $p_{T}$ cuts & 393607.37 & 86.93 & 330259.17 & 87.07 & 1297703.68 & 87.13 \\
$m_{\gamma\gamma} \in [105,160]$~GeV & 222795.22 & 56.60 & 187632.87 & 56.81 & 741243.50 & 57.12 \\
\midrule
Efficiency(\%) & \multicolumn{2}{c}{14.76} & \multicolumn{2}{c}{14.35} & \multicolumn{2}{c}{14.06} \\
\bottomrule
\end{tabular}
\end{table}


The Number of Data and MC Events in 168 \ifb after the full selection is summarised in Table ~\ref{tab:yield_compare}.

\begin{table}[!htbp]
\centering
\caption{Yields in 168 \ifb for data and mc sample after the Full selection}
\label{tab:yield_compare}
\begin{tabular}{lrrrr}
\toprule
Process & \multicolumn{3}{c}{Yield} &  \\
 & 2022 & 2023 & 2024 & Total Yield \\
\midrule
ggH & 1365.82 & 1151.91 & 4536.37 & 7054.10 \\
VBF & 106.23 & 90.11 & 356.46 & 552.80 \\
$W^+H$ & 18.86 & 16.04 & 63.78 & 98.68 \\
$W^-H$ & 13.54 & 11.47 & 45.41 & 70.42 \\
$q\bar{q}\to ZH$ & 18.42 & 15.59 & 61.88 & 95.89 \\
$gg\to ZH$ & 3.94 & 3.33 & 13.25 & 20.52 \\
$t\bar{t}H$ & 15.16 & 12.84 & 51.08 & 79.08 \\
$b\bar{b}H$ & 15.30 & 12.14 & 49.25 & 76.69 \\
$\gamma\gamma$+jets & 222796.52 & 187635.35 & 741411.61 & 1151843.48 \\
data & 280376 & 256835 & 995431 & 1532642 \\
\bottomrule
\end{tabular}
\end{table}


\subsection{Object selection and fiducial region definition at particle level}
\label{sec:fiducial}
In order to minimise the model dependence introduced by the extrapolation
from the phase space region selected by the kinematic and isolation criteria of
the selection, as well as by the detector acceptance to the phase space region
in which the cross-sections are measured, fiducial measurements are performed
in a phase space volume which is as close as possible to that selected by the
detector level criteria. The following particle level selection criteria are
applied.

Only stable final state particles are considered at particle level.
These are identified by requiring ${\texttt{status}=1}$ and
${\texttt{barcode} < 200,000}$, where the latter requirement
ensures that particles were not created by the GEANT simulation.
The four-momentum sum of these particles has a mass of \SI{13.6}{\TeV} by construction.
The type of each particle is represented by its PdgId.

\vspace{0.4cm}

\textbf{Photons} ~\hspace{0.2cm}~ From the TruthPhotons container,
  they are identified by requiring PdgId = 22.
  Must not be produced during hadronisation.
  This means that their parent should not have $|\text{PdgId}| \geq 111$.
  If the parent has $|\text{PdgId}| = 15$, corresponding to
  an intermediate $\tau$, or the same PdgId as the photon itself,
  corresponding to a final state radiative emission, then the
  PdgId of the grandparent is checked, and so on.


\vspace{0.4cm}

\textbf{Kinematic photon preselection} ~\hspace{0.2cm}~ Kinematic selections of $\pT > \SI{25}{\GeV}$ and $|\eta|<1.37$ or $1.52<|\eta|<2.37$ are applied to the selected photons.

\vspace{0.4cm}
\textbf{Diphoton system} ~\hspace{0.2cm}~ The two photons with the highest \pT are defined as the Higgs candidate system.
These photons are selected after the kinematic photon cuts have been applied.

\vspace{0.4cm}
\textbf{Isolated photons} ~\hspace{0.2cm}~ As in previous Higgs to diphoton cross-section measurements, a photon isolation requirement is applied at particle level to reduce model dependence in the measurement. The motivation for using particle level isolation and the method for choosing the cut is further discussed in Appendix~\ref{sec:truthiso}. The isolation energy, $E^\text{iso}_\text{T}$ of a particle-level photon is calculated as the transverse energy, \ET of the four-vector sum of all charged particles with $\pt>~\SI{1}{GeV}$ and $\Delta R < 0.2$ to the photon. The event is discarded if at least one of the two photons fails the requirement $E^\text{iso}_\text{T} <  0.05\pt$.
%\colorbox{yellow}{To be re-checked for Run 3.}

The `diphoton` fiducial region is defined by the presence of two isolated photons in the final state with transverse momenta greater than 35\% of the diphoton invariant mass for the leading photon and 25\% for the subleading photon.

The definitions of the additional event categories require additional objects passing the below particle level selections:

\begin{itemize}
\item  Leptons are defined from all electrons and muons that are not produced during hadronisation. The prompt leptons are dressed by adding the four-momenta of stable photons within $\Delta R < 0.1 $. Selected electrons (muons) are required to pass the kinematic selection of $p_{T}>15$~GeV and $|\eta|<2.47$ ($|\eta|< 2.7$ for muons). Electrons are rejected if they pass through the barrel-endcap transition region ($1.37 < |\eta| < 1.52$), or if their distance from a selected photon is less than $0.4$. No isolation requirement is applied.

\item Jets are defined by clustering all stable particles using the anti-$k_t$ algorithm with a radius parameter $R=0.4$. The clustering algorithm excludes prompt leptons and Higgs boson decay products. Selected jets are required to have $p_{T}>30$~GeV and $|y| <4.4$. Selected jets are required to be well separated from photons with $p_{T} > 15$~GeV ($\Delta R > 0.4$) and electrons ($\Delta R > 0.2$). Leading and subleading jets are defined as the ones with the largest and second-largest transverse momenta.

\item Jets are defined by clustering all stable particles using the anti-$k_t$ algorithm with a radius parameter $R=0.4$. The clustering algorithm excludes prompt leptons and Higgs boson decay products. Selected jets are required to have $p_{T}>30$~GeV and $|y| <4.4$. Selected jets are required to be well separated from photons with $p_{T} > 15$~GeV ($\Delta R > 0.4$) and electrons ($\Delta R > 0.2$).

\item $b$-jets are defined from selected central jets ($|\eta|<2.5$) if there is a $b$-hadron with $p_{T} >  5$~GeV within $\Delta R = 0.4$ of the jet axis.

\item Missing transverse momentum  is defined as the vector sum of the transverse momenta of all neutrinos that do not originate from the decays of hadrons.

\end{itemize}
