\section{Signal Yield Extraction}
\label{sec:signalyieldextraction}
 
The methodology to extract the signal yields follows Ref.~\cite{HIGG-2019-13}. 
For each distribution under study, the signal yields are extracted from a fit of a S+B PDF to the \mgg{} distribution. An example of such fit for the inclusive diphoton fiducial 
cross-section  is shown in Figure~\ref{fig:myy_fits_bin_by_bin_AsimovSB_Diphoton_constant}. The yields are extracted with {\ttfamily RooFit},
using {\ttfamily xmlAnaWorkspaceBuilder} and {\ttfamily quickFit} wrappers to build and fit the workspaces.

\begin{figure}[htb!]
 \begin{center}
 \includegraphics[width=0.5\textwidth]{figures/myy_fits/bin_by_bin/AsimovSB/Diphoton/constant/XSectionWS_Diphoton_constant_0.pdf}
 \end{center}
 \caption {An example of a simultaneous fit of the signal plus background PDF (full line) to pseudo-data (black points), 
 as used to extract the signal yield. The dashed like shows the background-only component.}
 \label{fig:myy_fits_bin_by_bin_AsimovSB_Diphoton_constant}
 \end{figure}

\subsection{Fit Objective}
%\label{ssec:fit_objective}
% largely from run2 int note 

The objective of the S+B fit is to maximise the likelihood:
\begin{align}
 \mathcal{L}(\mgg; \nsig, \nbkg) = \frac{ e^{-\nsigbkg} }{n!} \prod_j^{\nsigbkg} \bigg[ \nsig \, \mathcal{S}(m_{\gamma \gamma}^j; \theta_k) + \nbkg \, \mathcal{B}(m_{\gamma \gamma}^j)  \bigg]
 \label{eq:likelihood}
\end{align}
where \nsig is the fitted number of signal events, \nbkg is the fitted number of background events, $\nsigbkg = \nsig+ \nbkg$ 
is the mean value of the underlying Poisson distribution for the $n$ events, $\mgg^j$ is the diphoton invariant mass for event 
$j$, $\mathcal{S}(\mgg^j; \theta_k)$ and $\mathcal{B}(\mgg^j)$ are the signal and background probability distribution functions.

The product of the Poisson likelihood for each bin is then multiplied by the product of the nuisance parameter constrains:
\begin{align}
 \prod_k C_k\left(\theta_k;0,1\right).
 \label{eq:constraints}
\end{align}
In this way, the nuisance parameters are correlated between all bins in a given distribution.
This means that they are allowed to float in the fit, though deviations from zero are penalised by a reduction in the likelihood.
For the background, the functional form is determined by the spurious signal tests, and it's parameters are allowed to float in the fit. 

% From Run2. TODO LM : x-check 
%The constraint functions $C_k$ are Gaussians with a mean of $0$ and a width of $1$ for each of the (symmetrised) energy scale nuisance parameter $\theta_\text{ES}$, 
%as well as for the Higgs boson mass $m_H$ uncertainty. For the resolution uncertainties we use a Gaussian pdf together with an asymmetric response function: RooStats::HistFactory::FlexibileInterpVar, with upper variation as (central value) + (upper uncertainty) and
%lower variation as 1/(1+(lower uncertainty)/(central value)). Interpolation code 4 is used to make sure the transtion at NP=0 is smooth.


\subsection{Signal Yields in the Asimov Dataset}
\label{ssec:asimov}

The signal yield extraction is developed using so-called Asimov dataset, named after Isaac Asimov, who, in a short story Franchise, 
envisioned elections decided by a single most representative member of the electorate~\cite{Cowan:2010js}. For this analysis, the most representative 
sample is generated by: 
\begin{enumerate}
\item fitting the B-only PDF to the data \mgg{} side--band,
\item generating B pseudo-data from the PDF fitted in step 1.),
\item adding the signal according to the SM expectation.
\end{enumerate}

With such pseudo-data, the expectation is that in the S+B fits:
\begin{itemize}
\item the fitted PDF very closely matches the pseudo-data; note very small residuals in Figure~\ref{fig:myy_fits_bin_by_bin_AsimovSB_Diphoton_constant}. 
\item The nuisance parameters are expected to have very small pulls or constraints. 
\end{itemize}
The terms \textit{very closely} and \textit{very small} are used here instead of \textit{exactly} and \textit{zero}
to reflect the following feature, which has no impact on the final result. 
For the Asimov dataset, the background in step 2.) is generated from a B-only PDF,
and the signal in step 3.) from the S-only PDF. Whereas the signal yields are extracted from a fit of a S+B PDF.
In high precision checks, this difference would incur a small mis-match between the pseudo-data and the fitted PDF, 
or slight pulls or constraints on the nuisance parameters - these however have a negligible impact on the final result. 

\subsubsection{Residuals of the Fit to the Asimov Dataset}
%\label{ssec:residuals}  

The residuals of the fit to the Asimov data are shown in Figures~\ref{fig:myy_fits_bin_by_bin_AsimovSB_constant_residuals_1}, 
Figures~\ref{fig:myy_fits_bin_by_bin_AsimovSB_constant_residuals_2} for the MET, VBF, Lepton and ttH selections. In all cases 
the expectation that the residuals are small is confirmed across the \mgg{} range. 

\begin{figure}[htb!]
  \begin{center}
 \includegraphics[width=0.4\textwidth]{figures/myy_fits/bin_by_bin/AsimovSB/MET/constant/XSectionWS_MET_constant_0.pdf} 
    \includegraphics[width=0.4\textwidth]{figures/myy_fits/bin_by_bin/AsimovSB/VBF/constant/XSectionWS_VBF_constant_0.pdf}
  \end{center}
  \caption{Signal plus background PDF (full line) fits to the Asimov data (black points) for the MET and VBF selections. 
 The bottom panel show the residuals of the fit, defined as the difference between the 
 Asimov data and the fitted PDF. The dashed like shows the background-only component of the PDF.}
  \label{fig:myy_fits_bin_by_bin_AsimovSB_constant_residuals_1}
\end{figure}
\begin{figure}[htb!]
  \begin{center}
    \includegraphics[width=0.4\textwidth]{figures/myy_fits/bin_by_bin/AsimovSB/Lepton/constant/XSectionWS_Lepton_constant_0.pdf}  
    \includegraphics[width=0.4\textwidth]{figures/myy_fits/bin_by_bin/AsimovSB/ttH/constant/XSectionWS_ttH_constant_0.pdf}
  \end{center}
  \caption{Signal plus background PDF (full line) fits to the Asimov data (black points) for the Lepton and ttH selections. 
 The bottom panel show the residuals of the fit, defined as the difference between the 
 Asimov data and the fitted PDF. The dashed like shows the background-only component of the PDF.}
  \label{fig:myy_fits_bin_by_bin_AsimovSB_constant_residuals_2}
\end{figure}

For the differential cross-sections, the corresponding fits are shown for the first bin of each of the distributions in 
Figure~\ref{fig:myy_fits_bin_by_bin_AsimovSB_diff_1} for the diphoton observables \ptgg{} and \ygg{}, 
in Figure~\ref{fig:myy_fits_bin_by_bin_AsimovSB_diff_2} for the dijet observables \mjj{} and \dphijj{}, 
for \ptj and \Njets in Figure~\ref{fig:myy_fits_bin_by_bin_AsimovSB_diff_3} and for \Nlept and \Nbjets in 
Figure~\ref{fig:myy_fits_bin_by_bin_AsimovSB_diff_4}. In each of the figures, the expected small residuals are confirmed; 
this is also the case for other bins in the distributions, not shown here. 

\begin{figure}[htb!]
  \begin{center}
 \includegraphics[width=0.4\textwidth]{figures/myy_fits/bin_by_bin/AsimovSB/Diphoton/pT_yy/XSectionWS_Diphoton_pT_yy_0.pdf} 
 \includegraphics[width=0.4\textwidth]{figures/myy_fits/bin_by_bin/AsimovSB/Diphoton/yAbs_yy/XSectionWS_Diphoton_yAbs_yy_0.pdf}
  \end{center}
  \caption{Signal plus background PDF (full line) fits to the Asimov data (black points) for the first bin of the \ptgg{} and \ygg{}. 
 The bottom panel show the residuals of the fit, defined as the difference between the 
 Asimov data and the fitted PDF. The dashed line shows the background-only component of the PDF.}
  \label{fig:myy_fits_bin_by_bin_AsimovSB_diff_1}
\end{figure}

\begin{figure}[htb!]
  \begin{center}
 \includegraphics[width=0.4\textwidth]{figures/myy_fits/bin_by_bin/AsimovSB/Diphoton/m_jj_30/XSectionWS_Diphoton_m_jj_30_0.pdf} 
 \includegraphics[width=0.4\textwidth]{figures/myy_fits/bin_by_bin/AsimovSB/Diphoton/Dphi_j_j_30_signed/XSectionWS_Diphoton_Dphi_j_j_30_signed_0.pdf}
  \end{center}
  \caption{Same as Figure~\ref{fig:myy_fits_bin_by_bin_AsimovSB_diff_1} but for the dijet observables \mjj{} and \dphijj{}.}
  \label{fig:myy_fits_bin_by_bin_AsimovSB_diff_2}
\end{figure}

\begin{figure}[htb!]
  \begin{center}
 \includegraphics[width=0.4\textwidth]{figures/myy_fits/bin_by_bin/AsimovSB/Diphoton/pT_j1_30/XSectionWS_Diphoton_pT_j1_30_0.pdf} 
 \includegraphics[width=0.4\textwidth]{figures/myy_fits/bin_by_bin/AsimovSB/Diphoton/N_j_30/XSectionWS_Diphoton_N_j_30_0.pdf}
  \end{center}
  \caption{Same as Figure~\ref{fig:myy_fits_bin_by_bin_AsimovSB_diff_1} but for the \ptj{} and \Njets{} observables.}
  \label{fig:myy_fits_bin_by_bin_AsimovSB_diff_3}
\end{figure}

\begin{figure}[htb!]
  \begin{center}
 \includegraphics[width=0.4\textwidth]{figures/myy_fits/bin_by_bin/AsimovSB/Diphoton/N_lep_15/XSectionWS_Diphoton_N_lep_15_0.pdf} 
 \includegraphics[width=0.4\textwidth]{figures/myy_fits/bin_by_bin/AsimovSB/Diphoton/N_j_btag30/XSectionWS_Diphoton_N_j_btag30_0.pdf}
  \end{center}
  \caption{Same as Figure~\ref{fig:myy_fits_bin_by_bin_AsimovSB_diff_1} but for the \Nlept{} and \Nbjets{} observables.}
  \label{fig:myy_fits_bin_by_bin_AsimovSB_diff_4}
\end{figure}

Overall, the figures in this section confirm the Asimov dataset building and the maximum likelihood
fit of Equation~\ref{eq:likelihood} works as expected.

\subsubsection{Pulls of the Nuisance Parameters in the Asimov Dataset}
\label{ssec:pulls}

This section checks the second expectation of the Asimov dataset fits, that the nuisance parameters (Equation \ref{eq:constraints}) 
have no pulls or constraints. The plots are to be replaced by ranking plots later on.
%The pulls are defined as the difference between the fitted value and the nominal value of 0, divided by the uncertainty of the fitted value. 
%The constraints are defined as the uncertainty of the fitted value divided by the nominal uncertainty

In the Figure~\ref{fig:pull_bin_by_bin_AsimovSB_constant}, the pulls of the nuisance parameters in the Asimov dataset are shown for the
constant binning of the diphoton, MET, VBF, Lepton and ttH selections, with no observable pulls or constraints, as expected.

\begin{figure}[htb!]
  \begin{center}
 \includegraphics[height=0.355\textheight]{figures/results_wonky_format/bin_by_bin/AsimovSB/Diphoton/constant/pull_Diphoton_constant.png} 
\includegraphics[height=0.355\textheight]{figures/results_wonky_format/bin_by_bin/AsimovSB/MET/constant/pull_MET_constant.png}
\includegraphics[height=0.355\textheight]{figures/results_wonky_format/bin_by_bin/AsimovSB/VBF/constant/pull_VBF_constant.png}
\includegraphics[height=0.355\textheight]{figures/results_wonky_format/bin_by_bin/AsimovSB/Lepton/constant/pull_Lepton_constant.png}
\includegraphics[height=0.355\textheight]{figures/results_wonky_format/bin_by_bin/AsimovSB/ttH/constant/pull_ttH_constant.png}
  \end{center}
  \caption{Nuisance parameter pulls and constraints in the Asimov dataset for the constant binning of the 
  diphoton, MET, VBF, Lepton and ttH selections. The value of 0 and error bar of of 1 indicates no pull or constraint,
  as expected by the Asimov dataset construction. The \textit{Bias} terms correspond to the spurious signal parameters.}
  \label{fig:pull_bin_by_bin_AsimovSB_constant}
\end{figure}

In Figures~\ref{fig:pull_bin_by_bin_AsimovSB_diff_1}, \ref{fig:pull_bin_by_bin_AsimovSB_diff_2} and \ref{fig:pull_bin_by_bin_AsimovSB_diff_3}, 
the pulls of the nuisance parameters are shown for the differential observables. In all cases, no significant pulls or constraints are observed.
A very slight jitter is observed in the highest bins of the \ptgg{} distribution (Fig.~\ref{fig:pull_bin_by_bin_AsimovSB_diff_1}) and the \Nbjets{}=1 bin, which, as 
argued earlier, are likely due to small difference in the S+B PDF fit compared to the sum of the independent S and B PDFs. 

\begin{figure}[htb!]
  \begin{center}
 \includegraphics[height=0.4\textheight]{figures/results_wonky_format/bin_by_bin/AsimovSB/Diphoton/pT_yy/pull_Diphoton_pT_yy.png} 
\includegraphics[height=0.4\textheight]{figures/results_wonky_format/bin_by_bin/AsimovSB/Diphoton/yAbs_yy/pull_Diphoton_yAbs_yy.png}
  \end{center}
  \caption{Same as Figure~\ref{fig:pull_bin_by_bin_AsimovSB_constant} but for the \ptgg{} and \ygg{} distributions.
  The \textit{Bias} terms correspond to the spurious signal parameters - one per each bin of the distribution.}
  \label{fig:pull_bin_by_bin_AsimovSB_diff_1}
\end{figure}


\begin{figure}[htb!]
  \begin{center}
 \includegraphics[height=0.4\textheight]{figures/results_wonky_format/bin_by_bin/AsimovSB/Diphoton/m_jj_30/pull_Diphoton_m_jj_30.png}
\includegraphics[height=0.4\textheight]{figures/results_wonky_format/bin_by_bin/AsimovSB/Diphoton/Dphi_j_j_30_signed/pull_Diphoton_Dphi_j_j_30_signed.png}
\includegraphics[height=0.4\textheight]{figures/results_wonky_format/bin_by_bin/AsimovSB/Diphoton/pT_j1_30/pull_Diphoton_pT_j1_30.png}
  \end{center}
  \caption{Same as Figure~\ref{fig:pull_bin_by_bin_AsimovSB_diff_1} but for the \dphijj{}, \mjj{} and \ptj{} distributions.
    %% TODOLM: Dphi_j_j_30_signed is in fact yyjj ; error somewhere in the plotting code!
    Note that \dphijj{} shows the wrong observable; will be fixed. 
  }
  \label{fig:pull_bin_by_bin_AsimovSB_diff_2}
\end{figure}

\begin{figure}[htb!]
  \begin{center}
 \includegraphics[height=0.4\textheight]{figures/results_wonky_format/bin_by_bin/AsimovSB/Diphoton/N_j_30/pull_Diphoton_N_j_30.png}
\includegraphics[height=0.4\textheight]{figures/results_wonky_format/bin_by_bin/AsimovSB/Diphoton/N_lep_15/pull_Diphoton_N_lep_15.png}
\includegraphics[height=0.4\textheight]{figures/results_wonky_format/bin_by_bin/AsimovSB/Diphoton/N_j_btag30/pull_Diphoton_N_j_btag30.png}
  \end{center}
  \caption{Same as Figure~\ref{fig:pull_bin_by_bin_AsimovSB_diff_1} but for the  \Njets{}, \Nlept{} and \Nbjets{}
   distributions.}
  \label{fig:pull_bin_by_bin_AsimovSB_diff_3}
\end{figure}