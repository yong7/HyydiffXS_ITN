\section{Expected Results (Asimov Dataset)}
\label{sec:asimovresults}

Using the Asimov dataset described in Section \ref{ssec:asimov} expected cross-sections and their uncertainties are
extracted. This provides:
\begin{enumerate}
\item Checks of the signal yield extraction; do the cross-sections match the injected SM predictions?
\item Means to investigate systematic uncertainties. In particular, as the uncertainties experience no pulls and constraints in Asimov dataset fits, 
any unexpected trends should be traced to their source.  
\end{enumerate}
The results are shown for each of Bin-by-bin Corrections in Section \ref{ssec:asimovresults_bin_by_bin} (available now) and for
Matrix-Inversion Unfolding in Section \ref{ssec:asimovresults_mi} (to be filled once available).

\subsection{Expected Results Using Bin-by-bin Corrections}
\label{ssec:asimovresults_bin_by_bin}
 
\begin{figure}[htb!]
  \begin{center}
 \includegraphics[width=0.4\textwidth]{figures/results_wonky_format/bin_by_bin/AsimovSB/Diphoton/constant/plot_Diphoton_constant.png} 
 \includegraphics[width=0.4\textwidth]{figures/results_wonky_format/bin_by_bin/AsimovSB/VBF/constant/plot_VBF_constant.png}
  \end{center}
  \caption{Expected cross-sections from bin-by-bin unfolding, for the inclusive Diphoton (left) and VBF (right) selections. The 
  SM predictions are shown in a full line, and the expected result as filled circles. The error bars are shown 
for each of the total (statistics+systematics) and statistics-only uncertainties.}
  \label{fig:plot_bin_by_bin_AsimovSB_1}
\end{figure}

\begin{figure}[htb!]
  \begin{center}
 \includegraphics[width=0.32\textwidth]{figures/results_wonky_format/bin_by_bin/AsimovSB/MET/constant/plot_MET_constant.png} 
 \includegraphics[width=0.32\textwidth]{figures/results_wonky_format/bin_by_bin/AsimovSB/Lepton/constant/plot_Lepton_constant.png}
  \includegraphics[width=0.32\textwidth]{figures/results_wonky_format/bin_by_bin/AsimovSB/ttH/constant/plot_ttH_constant.png}
  \end{center}
  \caption{Same as Figure~\ref{fig:plot_bin_by_bin_AsimovSB_1} but for MET (left) and Lepton (middle) and ttH (right) selections.}
  \label{fig:plot_bin_by_bin_AsimovSB_2}
\end{figure}
  
Figures \ref{fig:plot_bin_by_bin_AsimovSB_1}--\ref{fig:plot_bin_by_bin_AsimovSB_diff_3} provide checks of 
point 1) \textit{do the cross-sections match the injected SM predictions?} Specifically, the cross-sections should match the SM prediction times 
the relevant correction factor, as described in Section \ref{ssec:asimov}. The cross-sections in all figures are shown without 
the \hgg{} branching ratio. 

Figures \ref{fig:plot_bin_by_bin_AsimovSB_1} and \ref{fig:plot_bin_by_bin_AsimovSB_2} show the results for the inclusive selections, 
matching the injected values of: 29.8 fb for Diphoton, 0.626 fb for VBF, 0.134 fb for MET, 0.271 fb for Lepton and 0.271 fb for the ttH region. 
Consequently, the values in the bottom panels of the figures, which show the ratio of the injected and extracted cross-sections, are equal to 1.
The Lepton and ttH regions happen to have the same cross-sections within the rounding uncertainty, and the sub-process composition has been checked 
to verify this is a numerical coincidence rather than a bug. Spefically, as expected the ttH region contributions are predominantlt from the ttH production, 
whereas the Lepton region also has large contributions from the WH and ZH production. 
% R2 vs R3 with BR 
%ggH & 0.06360321988522163 & 0.06780961838061474
%VBF & 0.001314668111586092 & 0.0014211190061389895
%MET & 0.0002939593517821063 & 0.00030333575875796256
%Lepton & 0.0005671117028774617 & 0.0006154765838685054
%ttH    & /                     & 0.0006151611831697103

The figures also show the expected uncertainty, as a total (statistics+systematics) and statistics-only uncertainties. In all cases, statistics 
uncertainty contribution is sizeable. In the current iteration of the plots done with h032 samples, only a simplified uncertainty scheme could be used, 
and all results currently lack the theory modelling uncertainties. Once these are included, the statistics and systematics uncertainties will become more 
comparable, especially for the Diphoton selection. 


Figures \ref{fig:plot_bin_by_bin_AsimovSB_diff_1}--\ref{fig:plot_bin_by_bin_AsimovSB_diff_3} 
show the results for the differential distributions. The bottom panels likewise confirm that the extracted cross-sections  match the injected ones.  
For the differential distributions, the binning has been designed to reach the expected significance of about 2-$\sigma$ in most bins, as detailed in 
Section \ref{sec:binningoptimisation}. This criterion means that all bins are dominated by the statistical uncertainty. This will remain the case also 
once a complete set of systematics uncertainties is taken into account. 

\begin{figure}[htb!]
  \begin{center}
 \includegraphics[width=0.4\textwidth]{figures/results_wonky_format/bin_by_bin/AsimovSB/Diphoton/pT_yy/plot_Diphoton_pT_yy.png} 
\includegraphics[width=0.4\textwidth]{figures/results_wonky_format/bin_by_bin/AsimovSB/Diphoton/yAbs_yy/plot_Diphoton_yAbs_yy.png}
  \end{center}
  \caption{Expected cross-sections from bin-by-bin unfolding for the inclusive Diphoton selection and the \ptgg{} and \ygg{} distributions.
   The 
  SM predictions are shown in a full line, and the expected result as filled circles. The error bars are shown 
for each of the total (statistics+systematics) and statistics-only uncertainties.
  }
  \label{fig:plot_bin_by_bin_AsimovSB_diff_1}
\end{figure}

\begin{figure}[htb!]
  \begin{center}
 \includegraphics[width=0.32\textwidth]{figures/results_wonky_format/bin_by_bin/AsimovSB/Diphoton/m_jj_30/plot_Diphoton_m_jj_30.png}
\includegraphics[width=0.32\textwidth]{figures/results_wonky_format/bin_by_bin/AsimovSB/Diphoton/Dphi_j_j_30_signed/plot_Diphoton_Dphi_j_j_30_signed.png}
\includegraphics[width=0.32\textwidth]{figures/results_wonky_format/bin_by_bin/AsimovSB/Diphoton/pT_j1_30/plot_Diphoton_pT_j1_30.png}
  \end{center}
  \caption{Same as Figure~\ref{fig:plot_bin_by_bin_AsimovSB_diff_1} but for the \dphijj{}, \mjj{} and \ptj{} distributions.}
  \label{fig:plot_bin_by_bin_AsimovSB_diff_2}
\end{figure}

\begin{figure}[htb!]
  \begin{center}
 \includegraphics[width=0.32\textwidth]{figures/results_wonky_format/bin_by_bin/AsimovSB/Diphoton/N_j_30/plot_Diphoton_N_j_30.png}
\includegraphics[width=0.32\textwidth]{figures/results_wonky_format/bin_by_bin/AsimovSB/Diphoton/N_lep_15/plot_Diphoton_N_lep_15.png}
\includegraphics[width=0.32\textwidth]{figures/results_wonky_format/bin_by_bin/AsimovSB/Diphoton/N_j_btag30/plot_Diphoton_N_j_btag30.png}
  \end{center}
  \caption{Same as Figure~\ref{fig:plot_bin_by_bin_AsimovSB_diff_1} but for the  \Njets{}, \Nlept{} and \Nbjets{}
   distributions.}
  \label{fig:plot_bin_by_bin_AsimovSB_diff_3}
\end{figure}


The composition of the total uncertainty for each of the cross-sections can be studies by peforming the signal extraction fits with 
the family of nuisance parameters fixed of allowed to float. For the inclusive fiducial region selections, the results are shown 
for the Diphoton and VBF selections in Figure~\ref{fig:fraction_sys_errorgroup_bin_by_bin_AsimovSB_1} 
and  MET, Lepton and ttH  selections in Figre~\ref{fig:fraction_sys_errorgroup_bin_by_bin_AsimovSB_2}. 
The spurious signal is not visible in the plots, because the injected contributions (taken from Run2 for now) 
are of the order of magnitude $|max(S)/Sref|\sim 0.05$\% for the inclusive selections. With the current uncertainty scheme, 
the luminosity uncertainty taken as 2\% is the single largest contribution to the systematics uncertainty in all but the ttH region. 

\begin{figure}[htb!]
  \begin{center}
 \includegraphics[width=0.4\textwidth]{figures/results_wonky_format/bin_by_bin/AsimovSB/Diphoton/constant/fraction_sys_errorgroup_bin_by_bin_Diphoton_constant.png} 
 \includegraphics[width=0.4\textwidth]{figures/results_wonky_format/bin_by_bin/AsimovSB/VBF/constant/fraction_sys_errorgroup_bin_by_bin_VBF_constant.png}
  \end{center}
  \caption{Fractional uncertainty composition for the inclusive Diphoton (left) and VBF (right) selections. The left axis shows the fractional uncertainty 
with respect to the total systematics uncertainty. The right axis shows the total expected relative uncertainty, also including the statistical uncertainty. 
The display does not really work well and will be updated - the take-away from this right axis is that the total uncertainty is $\sim$ 8\% and 22\% for the 
Diphoton and VBF selections respectively.}
  \label{fig:fraction_sys_errorgroup_bin_by_bin_AsimovSB_1}
\end{figure}

\begin{figure}[htb!]
  \begin{center}
 \includegraphics[width=0.32\textwidth]{figures/results_wonky_format/bin_by_bin/AsimovSB/MET/constant/fraction_sys_errorgroup_bin_by_bin_MET_constant.png} 
 \includegraphics[width=0.32\textwidth]{figures/results_wonky_format/bin_by_bin/AsimovSB/Lepton/constant/fraction_sys_errorgroup_bin_by_bin_Lepton_constant.png}
  \includegraphics[width=0.32\textwidth]{figures/results_wonky_format/bin_by_bin/AsimovSB/ttH/constant/fraction_sys_errorgroup_bin_by_bin_ttH_constant.png}
  \end{center}
  \caption{Same as Figure~\ref{fig:fraction_sys_errorgroup_bin_by_bin_AsimovSB_1} but for MET (left), Lepton (middle) and ttH (right) selections.}
  \label{fig:fraction_sys_errorgroup_bin_by_bin_AsimovSB_2}
\end{figure}

Figures~\ref{fig:fraction_sys_errorgroup_bin_by_bin_bin_by_bin_AsimovSB_diff_1}, \ref{fig:fraction_sys_errorgroup_bin_by_bin_bin_by_bin_AsimovSB_diff_2} and 
\ref{fig:fraction_sys_errorgroup_bin_by_bin_bin_by_bin_AsimovSB_diff_3} show the uncertainty composition for the differential observables. For these, the spurions 
signal is typically the largest single contribution to the systematic uncertainty. 
TODO LM: check what is going on with mjj in Fig \ref{fig:fraction_sys_errorgroup_bin_by_bin_bin_by_bin_AsimovSB_diff_2} - only has 2 bins. 

\begin{figure}[htb!]
  \begin{center}
 \includegraphics[width=0.4\textwidth]{figures/results_wonky_format/bin_by_bin/AsimovSB/Diphoton/pT_yy/fraction_sys_errorgroup_bin_by_bin_Diphoton_pT_yy.png} 
\includegraphics[width=0.4\textwidth]{figures/results_wonky_format/bin_by_bin/AsimovSB/Diphoton/yAbs_yy/fraction_sys_errorgroup_bin_by_bin_Diphoton_yAbs_yy.png}
  \end{center}
  \caption{inclusive Diphoton selection and the \ptgg{} and \ygg{} distributions.}
  \label{fig:fraction_sys_errorgroup_bin_by_bin_bin_by_bin_AsimovSB_diff_1}
\end{figure}

\begin{figure}[htb!]
  \begin{center}
 \includegraphics[width=0.32\textwidth]{figures/results_wonky_format/bin_by_bin/AsimovSB/Diphoton/m_jj_30/fraction_sys_errorgroup_bin_by_bin_Diphoton_m_jj_30.png}
\includegraphics[width=0.32\textwidth]{figures/results_wonky_format/bin_by_bin/AsimovSB/Diphoton/Dphi_j_j_30_signed/fraction_sys_errorgroup_bin_by_bin_Diphoton_Dphi_j_j_30_signed.png}
\includegraphics[width=0.32\textwidth]{figures/results_wonky_format/bin_by_bin/AsimovSB/Diphoton/pT_j1_30/fraction_sys_errorgroup_bin_by_bin_Diphoton_pT_j1_30.png}
  \end{center}
  \caption{Same as Figure~\ref{fig:fraction_sys_errorgroup_bin_by_bin_bin_by_bin_AsimovSB_diff_1} but for the \dphijj{}, \mjj{} and \ptj{} distributions.}
  \label{fig:fraction_sys_errorgroup_bin_by_bin_bin_by_bin_AsimovSB_diff_2}
\end{figure}

\begin{figure}[htb!]
  \begin{center}
 \includegraphics[width=0.32\textwidth]{figures/results_wonky_format/bin_by_bin/AsimovSB/Diphoton/N_j_30/fraction_sys_errorgroup_bin_by_bin_Diphoton_N_j_30.png}
\includegraphics[width=0.32\textwidth]{figures/results_wonky_format/bin_by_bin/AsimovSB/Diphoton/N_lep_15/fraction_sys_errorgroup_bin_by_bin_Diphoton_N_lep_15.png}
\includegraphics[width=0.32\textwidth]{figures/results_wonky_format/bin_by_bin/AsimovSB/Diphoton/N_j_btag30/fraction_sys_errorgroup_bin_by_bin_Diphoton_N_j_btag30.png}
  \end{center}
  \caption{Same as Figure~\ref{fig:fraction_sys_errorgroup_bin_by_bin_bin_by_bin_AsimovSB_diff_1} but for the  \Njets{}, \Nlept{} and \Nbjets{}
   distributions.}
  \label{fig:fraction_sys_errorgroup_bin_by_bin_bin_by_bin_AsimovSB_diff_3}
\end{figure}

Finally, for the differential distributions, the correlations between the bins of the extracted cross-sections can be checked. 
The bin-by-bin corrections used in this section neglects the migration between the bins, 
and therefore the results are only expected to be close to more sophisticated unfolding methods, 
such as the matrix inversion unfolding described in Section \ref{ssec:asimovresults_mi}, 
for observables with roughly diagonal correlation 
matrices, such as \ptgg{} and \ygg{}. For observables with non-diagonal correlation matrices, such as jet-related observables, 
the extracted correlations are instead not representative of what's expected from the matrix inversion unfolding.
Due to these caveats, the correlations are only shown for two representative distributions in Figures \ref{fig:correlation_bin_by_bin_AsimovSB_diff_1}: 
\ygg{} with rouhgly diagonal correlation matrix and Njets with non-diagonal correlation matrix. 

\begin{figure}[htb!]
  \begin{center}
 \includegraphics[width=0.45\textwidth]{figures/results_wonky_format/bin_by_bin/AsimovSB/Diphoton/yAbs_yy/correlation_Diphoton_yAbs_yy_full.png}
 \includegraphics[width=0.45\textwidth]{figures/results_wonky_format/bin_by_bin/AsimovSB/Diphoton/N_j_30/correlation_Diphoton_N_j_30_full.png}
  \end{center}
  \caption{Correlations between the bins of the extracted cross-sections for the \ptgg{} (left) and Njets (right) distributions.}
  \label{fig:correlation_bin_by_bin_AsimovSB_diff_1}
\end{figure}

\subsection{Expected Results Using Matrix-Inversion Unfolding}
\label{ssec:asimovresults_mi}