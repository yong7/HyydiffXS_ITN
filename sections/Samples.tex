\section{Data and simulation samples}
\label{S. Data and MC Samples}
The analysis is performed on data and simulation samples reconstructed with
Athena release 23.
The input are HIGG1D1 derivations produced with AthDerivation cache 25.0.22;
the corresponding p-tags are p6456 (data), p6599 (signal MC), p6453/p6781 (background MC).
The full list of input DAODs is given in appendix~\ref{app:DAODs}.

The HIGG1D1 derivation applies a looser preselection than the one applied
offline.
In particular, events are required to pass the logical OR of various diphoton
triggers and to contain at least two offline photon candidates.

All samples are then analysed with the HGamCore analysis framework~\cite{HGamCore}.
Mini xAODs (MxAODs) are produced by the HGamma group and stored in /eos.
The 'h032' tag was used for the MxAOD production, on top
of AnalysisBase 25.2.43.
At this stage all particle candidates are calibrated, data and simulation
corrections and weights are computed and applied or saved in the MxAOD,
and a few extra variables useful for the analysis are calculated.
A preselection is performed, that requires the following:
\begin{itemize}
\item Duplicate events are removed
\item Data events must pass the good runs lists requirement (good data quality)
\item Events must be collected with at least one of several diphoton triggers
  (including the one used by this analysis).
\item Data events must pass detector quality flags.
  Events with data integrity errors in the calorimeters and
  incomplete events where some detector information is missing are rejected.
\item The event must have at least one primary vertex candidate
\item There must be at least two photon candidates passing loose identification
  requirements based on their shower shapes in the electromagnetic calorimeter
  and the leakage in the hadronic calorimeter, and not passing through
  regions of the calorimeter with cells connected to dead high-voltage lines.
\end{itemize}

Special MxAODs (``DetailedNoSkim'') without any selection applied
are produced for the signal MC samples that are used to compute
the response matrices needed for the unfolding.

The MxAODs are a single format common to several Higgs to diphoton analyses,
such as the measurement of Higgs boson couplings and simplified template
cross sections, search for Higgs bosons produced in association with dark
matter and so on.
For this reason they are rather large, containing many more variables (several
hundreds) than needed for this analysis.
For this reason, MxAODs are further skimmed into micro-xAODs (uxAODs) of
smaller size, with only the needed variables (at detector and particle level)
and only the events of interest (typically, the full selection is applied for
data and background MC uxAODs, while for signal MC uxAODs events that pass
the full selection or are in the fiducial volume are kept).

\subsection{Data samples}
\label{ssec:samples:data}
This analysis is based on partial run3 $pp$ collision data collected by ATLAS detector at central-of-mass energy $\sqrt{s}$ = 13.6 TeV between 2022 and 2024.After data quality requirements are applied to ensure that all the detector components are in good working condition The total corresponding integrated luminosity is 168 \ifb.The corresponding GRLs for each year are summarized in Table~\ref{tab:lumi_grl}.

Events are selected if thet pass either a diphoton trigger or single photon trigger\cite{egamma_trigger}.The diptoton trigger has a transverse energy thresholds of 35GeV and 25GeV for the leading and subleading photon candicates respectively,with medium photon identification based on calorimeter shower shape variables.The single-photon photon trigger require the transverse enegry of the leading photon greater than 140 GeV and pass the loose photon identification.On average,the trigger efficiency is greater than 98\% for the events pass the diphoton event selection.However,The additional improvement from the single photon trigger is less than 1\% . Table ~\ref{tab:trig} summarized the HLT diphoton trigger for each year. 

\begin{table}[!htbp]
  \centering
  \scriptsize
  \begin{tabular}{clc}
    %\hline\hline
    \toprule
    Year & GRL & Luminosity (\ifb) \\
    \hline
    2022 & \texttt{\tiny data22\_13p6TeV.periodAllYear\_DetStatus-v109-pro28-04\_MERGED\_PHYS\_StandardGRL\_All\_Good\_25ns\_ignore\_TRIGMUO\_TRIGLAR.xml} & 31.39 \\
   %\hline
    2023 & \texttt{\tiny data23\_13p6TeV.periodAllYear\_DetStatus-v110-pro31-06\_MERGED\_PHYS\_StandardGRL\_All\_Good\_25ns\_ignoreTRIG\_JETCTPIN.xml}  & 27.23\\
    %\hline
    2024  & \texttt{\tiny physics\_25ns\_data24.xml} & 109.4\\
    \hline
    Total  & -  & 168\\    
    \bottomrule
  \end{tabular}
  \caption{The corresponding GRL and luminosity for each year}
  \label{tab:lumi_grl}
\end{table} 

\begin{table}[htbp]
    \centering
    \begin{tabular}{c|c}
      \toprule
        Year & trigger \\
        \hline
        2022 & \texttt{HLT\_g35\_medium\_g25\_medium\_L12EM20VH} \\
        \hline
        2023 & \texttt{HLT\_g35\_medium\_g25\_medium\_L12eEM24L} \\
        \hline
        2024 & \texttt{HLT\_g35\_medium\_g25\_medium\_L12eEM24L} \\
        \bottomrule
    \end{tabular}
    \caption{The di-photon trigger used by this analysis for each year.}
    \label{tab:trig}
\end{table}

\subsection{Simulation samples}
\label{ssec:samples:mc}

Simulated event samples are used to estimate the expected signal yield,
to optimize the binning of the differential cross section measurements,
to model the signal diphoton invariant mass for the fit to the data used to
determine the signal and background yields,and to estimate the correction factors to unfold the detector-level yields to the particle-level ones.
They are also used to determine the relative fractions of different
type of backgrounds and build, with the help of data control regions,
a background-only template that is used to choose an analytic function
to model the diphoton invariant mass distribution of the background and to
estimate the corresponding systematic uncertainty on the signal yield
determined from the signal+background fit to the data.

The event generators used to generate the nominal signal and background events,
the PDFs used for the matrix element generation, and the number of generated
events are summarized in Table~\ref{tab:samples:nominal_mc}.
The generated signal events were then passed to the full ATLAS detector
simulation based on Geant 4, while generated QCD diphoton events were passed
through a fast simulation of the ATLAS detector to reduce the amount of CPU
needed given the large numbers of events generated.

The matrix element for the nominal signal samples are generated with \POWHEGBOX[v2]~\cite{Nason:2009ai,Alioli:2010xd,Nason:2004rx,Frixione:2007vw}.The simulation achieved NNLO accuracy for arbitrary inclusive \(gg\to H\) observables by reweighting the Higgs boson rapidity spectrum in \textsc{Hj}-\MINLO~\cite{Hamilton:2012np,Campbell:2012am,Hamilton:2012rf} to that of HNNLO~\cite{Catani:2007vq}.The VBF, $t\bar{t}H$ and $b\bar{b}H$ processes were simulated at next-to-leading-order (NLO) accuracy in QCD.The WH and ZH were simulated by following the \MINLO~\cite{Hamilton:2012np} approach,while the ggZH was only computed at leading order(LO) in QCD.

All the signal sample matrix element use the PDFLHC21 PDF set and the generated events are showered with Pythia8 parton shower with A14 tune in which the $H \to \gamma\gamma $ is simulated.In contrast to full Run2 sample.The Dalitz decay of $H \to \gamma\gamma^* \to \gamma ff$ is not considered in the generation, where $\gamma^*$ is an off-shell photon and $f$ is any charged fermion.
% Samples of simulated $tH$ events are also available; however, we estimated
% the contribution of this production mode to the analysis to be negligible
% and therefore they won't be used in the final analysis results (they are
% listed here for completeness).

The continuum diphoton background samples are generated with Sherpa 2.2.14.in this set-up, NLO-accurate matrix elements for up to one parton, and LO-accurate matrix elements for up to three partons were calculated with the Comix~\cite{Gleisberg:2008fv} and
\OPENLOOPS~\cite{Buccioni:2019sur,Cascioli:2011va,Denner:2016kdg} libraries. They were matched with the \SHERPA parton shower~\cite{Schumann:2007mg} using the \MEPSatNLO prescription~\cite{Hoeche:2011fd,Hoeche:2012yf,Catani:2001cc,Hoeche:2009rj}
with a dynamic merging cut~\cite{Siegert:2016bre} of \qty{10}{\GeV}.
Photons were required to be isolated according to a smooth-cone isolation
criterion~\cite{Frixione:1998jh}. Samples were generated using the
\NNPDF[3.0nnlo] PDF set~\cite{Ball:2014uwa}, along with the dedicated set of tuned
parton-shower parameters developed by the \SHERPA authors.


\begin{table}[!htbp]
  \centering
  \caption{Nominal simulated signal and background event samples}
  \label{tab:samples:nominal_mc}
\resizebox{\textwidth}{!}{    
  \begin{tabular}{lccccrrr}
    %\hline
    %\hline
    \toprule
    Process & DSID & Generator (ME+PS) & PDF (ME) & PDF+Tune & \multicolumn{3}{c}{Events} \\
            &      &                   &          &          & MC23a & MC23d & MC23e \\
    %\hline
    \midrule
    ggF              & 602421 & Powheg NNLOPS + Pythia & PDF4LHC21 & A14NNPDF23LO & 4.46M  & 4.46M  & 11.45M  \\
    VBF              & 601482 & Powheg+Pythia          & PDF4LHC21 & A14NNPDF23LO & 2.19M  & 2.19M  & 5.45M  \\
    $W^+H$           & 601484 & Powheg MINLO+Pythia    & PDF4LHC21 & A14NNPDF23LO & 0.30M & 0.30M & 0.61M \\
    $W^-H$           & 601483 & Powheg MINLO+Pythia    & PDF4LHC21 & A14NNPDF23LO & 0.29M & 0.30M & 0.60M \\ 
    $q\bar{q}\to ZH$ & 601523 & Powheg MINLO+Pythia    & PDF4LHC21 & A14NNPDF23LO & 0.59M & 0.59M & 1.25M \\
    $gg\to ZH$       & 601522 & Powheg+Pythia          & PDF4LHC21 & A14NNPDF23LO &   0.1M &   0.1M &   0.2M \\
    $t\bar{t}H$      & 602422 & Powheg+Pythia          & PDF4LHC21 & A14NNPDF23LO & 0.79  & 0.78M  & 1.98M  \\
    $b\bar{b}H$      & 601710 & Powheg+Pythia          & PDF4LHC21 & A14NNPDF23LO & 0.20M & 0.20M & 0.5M \\
    %$tHjb$           & 545636 & MG5\_aMC@NLO+Pythia   & PDF4LHC21 & A14NNPDF23LO & 0.49M & 0.46M & 1.69M \\
    %$tWH$            & 545639 & MG5\_aMC@NLO+Pythia   & PDF4LHC21 & A14NNPDF23LO &  0.49M & 0.49M & 1.74M \\
    \hline
    $\gamma\gamma$+jets, $m_{\gamma\gamma} \in$ 90--175~GeV & 700980 & Sherpa (ME@NLO+PS) & NNPDF3.0NNLO & Sherpa & 227M & 215M & 703M \\
    %\hline
    %\hline
    \bottomrule
  \end{tabular}
}
\end{table} 






Alternative signal samples were also generated in order to estimate
uncertainties related to the modelling of the parton shower or of
the matrix element and in particular of extra jet radiation in gluon fusion.
For the estimation of the uncertainties related to the modelling of
the parton shower, for the ggF, VBF, $VH$, $tH$ ($ttH$) samples the
same events from the matrix-element generator of the nominal signal
samples were showered with Herwig 7.2.3 instead of Pythia
(see Table~\ref{tab:samples:herwig_mc}),using the H7UE set
of tuned parameters~\cite{Bellm:2015jjp} and the \MMHT[lo] PDF set~\cite{Harland-Lang:2014zoa}.


%%Due to some technical issues, the VBF sample only includes the nominal
%%event weight, calculated with the NNPDF3.0 NLO (CHECK) PDF.
%For the estimation of the uncertainties related to the matrix
%element calculation, alternative samples were produced with MadGraph5
%interfaced to Pythia8 (see Table~\ref{tab:samples:mg5_mc}).
% $VH$ samples were also produced, but they only contain leptonic
% decays of the $V$ boson and are therefore not usable. The reason is
% that the MadGraph5 VBF sample was supposed to include also $VH$
% events with hadronic $V$ boson decays, but due to some mistake in
% the configuration of the central production of those samples, this
% part is missing from the VBF sample.

\begin{table}[!htbp]
  \centering
  \caption{Alternative simulated signal samples showered with Herwig 7.}
  \label{tab:samples:herwig_mc}
\resizebox{\textwidth}{!}{    
  \begin{tabular}{lccccrrr}
    \toprule
    Process & DSID & Generator (ME+PS) & PDF (ME) & PDF+Tune & \multicolumn{3}{c}{Events} \\
            &      &                   &          &          & MC23a & MC23d & MC23e \\
    \midrule
    ggF              & 604315 & Powheg NNLOPS + Herwig & PDF4LHC21 & H7UE & 4.46M  & 4.45M  & 11.45M  \\
    VBF              & 604318 & Powheg+Herwig          & PDF4LHC21 & H7UE & 2.2M  & 2.19M  & 5.45M  \\
    $W^+H$           & 604312 & Powheg MINLO+Herwig    & PDF4LHC21 & H7UE & 0.29M & 0.30M & 0.61M \\
    $W^-H$           & 604311 & Powheg MINLO+Herwig    & PDF4LHC21 & H7UE & 0.29M & 0.30M & 0.60M \\ 
    $q\bar{q}\to ZH$ & 604313 & Powheg MINLO+Herwig    & PDF4LHC21 & H7UE & 0.59M & 0.59M & 1.23M \\
    $gg\to ZH$       & 604316 & Powheg+Herwig          & PDF4LHC21 & H7UE &   0.1M &   0.1M &   0.2M \\
    $t\bar{t}H$      & 604317 & Powheg+Herwig          & PDF4LHC21 & H7UE & 0.79  & 0.78M  & 1.97M  \\
    $b\bar{b}H$      & 604314 & Powheg+Herwig          & PDF4LHC21 & H7UE & 0.20M & 0.20M & 0.5M \\
    %$tHjb$           & 545636 & MG5\_aMC@NLO+Herwig   & PDF4LHC21 & H7UE & 0.49M & 0.46M & 1.69M \\
    %$tWH$            & 545639 & MG5\_aMC@NLO+Herwig   & PDF4LHC21 & H7UE &  0.49M & 0.49M & 1.74M \\
    \bottomrule
  \end{tabular}
}
\end{table}


In all generated signal samples (nominal and alternative) the Higgs
boson mass is assumed to be 125.0~GeV and its natural width is
set to $\Gamma_H = 4.07$~MeV.

The effect of multiple interactions in the same and neighbouring bunch crossings (\pileup) was modelled by overlaying~\cite{SIMU-2020-01} the simulated hard-scattering event with inelastic proton--proton (\(pp\)) events generated from a mix of  \EPOS[2.0.1.4]~\cite{Werner:2005jf} and \PYTHIA[8.310]~\cite{Bierlich:2022pfr}.
The \EPOS events were generated with the \EPOS LHC tune~\cite{Pierog:2013ria} 
and the \PYTHIA events with the A3 tune~\cite{ATL-PHYS-PUB-2016-017} 
and the \NNPDF[2.3lo]~\cite{Ball:2012cx} set of parton distribution functions (PDF).
\PYTHIA pileup events include either a high transverse momentum (\pt) jet, 
a prompt photon, or a lepton from a $b$-hadron decay, while \EPOS was filtered to simulate all remaining pileup events in the overlay sample.The individual simulations were first reweighted to ensure a smooth connection across jet \pt then
the combination reweighted to match the distribution of the actual number of interactions per bunch crossing (\(\mu\)) measured in data.

Event weights are applied to the simulation to account for data/MC
efficiency scale factors different from one, as determined from
control samples of various particles.
Energies and momentum corrections and smearing are applied
so that the momentum scales and resolution in the simulation
agree with those in data.
The corrections provided by the various performance groups in
AnalysisBase 25.2.43. were used.

The sum of the generator-level event weights of all generated events,
the luminosity of the data to be matched and the Higgs production
cross sections and branching ratio to diphotons are used to determine
the normalization factor for each simulated signal sample.




\begin{table}[!htbp]
  \centering
  \caption{SM predictions for the Higgs boson production cross section times diphoton branching ratio.}
  \label{tab:samples:signal_xsecbr}
  \scriptsize
  \begin{tabular}{lrll}
    %\hline
    %\hline
    \toprule
    Production mode & $\sigma\times BR$ [fb] & accuracy (of $\sigma$) & total events produced in 139~\ifb \\
    %\hline
    \midrule
    ggF              & 118.4259 & N3LO(QCD), NLO(EW)         & 19896 \\
    VBF              & 9.25025  & approx NNLO(QCD), NLO (EW) &  1554 \\
    $W^+H$           & 2.01349  & NNLO(QCD), NLO(EW)         &   338 \\
    $W^-H$           & 1.285728  & NNLO(QCD), NLO(EW)        &   216 \\
    $q\bar{q}\to ZH$ & 1.830301 & NNLO(QCD), NLO(EW)         &   307 \\
    $gg \to ZH$      & 0.308493 & NLO(QCD), NLO(EW)          &    52 \\
    $t\bar{t}H$      & 1.1291176 & NLO(QCD), NLO(EW)         &   190 \\
    $b\bar{b}H$      & 1.193339  & NLO(QCD), NLO(EW)         &   200 \\
    %$tHjb$          & 0.18602  & NLO(QCD)                   &    31 \\
    %$tHW$           & 0.039044 & NLO(QCD)                   &     7 \\
    % sigma*BR from https://gitlab.cern.ch/atlas-hgam-sw/HGamCore/blob/master/HGamAnalysisFramework/data/MCSamples.config#L368
    \hline
    Total            & 135.43   & -                          & 22752 \\
%%    Total w $tH$   & 135.65   & -                          & 22789 \\
    %\hline
    %\hline
    \bottomrule
  \end{tabular}
\end{table}

\FloatBarrier
\clearpage

