\section{Uncertainties}
\label{sec:uncertainties}

The main systematic uncertainties affecting the differential cross-section measurements can be divided into two sources, theoretical and experimental.
The theoretical uncertainties include the QCD scale, parton density functions, the value of the strong coupling constant and branching ratios.
Most of these uncertainties can cause migrations between different bins and are therefore computed as uncertainties on the efficiency and the acceptance for each observable.

\textcolor{blue}{This needs to be revisited when we update the results to h033} Experimental uncertainties act both on the signal yield and shape. 
The main uncertainties on the yield originate from the spurious signal, the measured integrated luminosity, trigger efficiency and Higgs mass, pileup-reweighting, the selection of the primary vertex, as well as photon identification and isolation. 
Uncertainties on event migration include jet uncertainties, lepton efficiency and ID, $E_T^{miss}$, flavour tagging and pileup-reweighting. 
Finally, uncertainties on the signal shape are introduced by the photon energy resolution and scale.

\subsection{Experimental Uncertainties}
\label{sec:experimentaluncertainties}

\subsubsection{Signal shape uncertainties}
\label{sssec:signal_shape_syst}

The photon energy scale (PES) and the photon energy resolution (PER) variations act both on the signal shape and on the overall yields. 
Their effect is included in the fit to data as response functions on $\mu_\text{CB}$ and $\sigma_\text{CB}$, respectively. 
The remaining parameters of the Double Sided Crystal Ball function used to model the signal are kept constant. 
These systematic variations are extracted for each analysis category and are treated as fully correlated among categories. 
All categories in this analysis are expected to be more sensitive to the PER than to the PES, since the change in the signal width affects the S/B ratio. 

The uncertainties are computed from MC signal samples with all production processes merged according to the predicted SM fractions, using the following techniques:
\begin{itemize}
	\item for the \textbf{scale}, the ratio-of-means technique is used: the mean \myy is computed for the nominal and $\pm 1 \sigma$ varied distributions. Then, the systematic uncertainty applied to $\mu_\text{CB}$ is calculated as
	\begin{equation}
	\delta \mu_\text{CB}^{\pm 1 \sigma} = \frac{\left\langle m_{\gamma\gamma}^{\pm 1 \sigma}\right\rangle}{\left\langle m_{\gamma\gamma}^\text{nom}\right\rangle}-1.
	\end{equation}
	Scale uncertainties are implemented in the fit to the data with a Gaussian constraint using the $+1\sigma$ variation, because of the high degree of symmetry observed in the uncertainties. All the constraints are multiplied together and applied to $\mu_\text{CB}$ as a response function. The shift in the signal peak is usually below 0.3\%, depending on the category under consideration, with the greatest contribution usually being the L2 Gain systematic. Plots for each category are shown in \textcolor{blue}{Plots to be included using h033};
	\item for the \textbf{resolution}, the ratio of inter-quartile ranges is used: the inter-quartile range is computed as $S = \text{CDF}^{-1}(75\%) - \text{CDF}^{-1}(25\%)$, where CDF is the cumulative distribution function of the \myy nominal and varied distributions. Then, the uncertainty is evaluated as
	\begin{equation}
	\delta \sigma_\text{CB}^{\pm 1 \sigma} = \frac{S^{\pm 1 \sigma}}{S^\text{nom}}-1.
	\end{equation}
	Resolution uncertainties are implemented in the fit with an asymmetric constraint, to take into account differences in the up/down components. All the resolution constraints are multiplied together and applied to $\sigma_\text{CB}$ as a response function. The effect of the PER uncertainty on the signal resolutions is typically between 1 and 8\%. \textcolor{blue}{Plots to be included using h033}.
\end{itemize}
%The list of the systematic uncertainties could be found in the \appendixname~\ref{app:ExpSysShape}\\

%\paragraph{Correlation model} Two models are provided by the calibration group: the first with 2 nuisance parameters (one for the scale and one for the resolution uncertainties), the latter with 80 systematics sources, 71 of which dedicated to scale uncertainties. As in the previous analysis, we will be able to constrain the 1NP model, so we have to go for the full one: in particular, we use a full decorrelated model for the resolution (9 NPs) and a merged one for the scale (40 NPs, since we are not so sensitive to scale variations): the NPs related to the material in front of the calorimeter (MATCALO and MATCRYO), to the presampler (PS) and to S12, which are divided in many $\eta$ bins, are sum together in two contributions each, one bin for the barrel and one for the endcap region.
\subsubsection{Signal yield uncertainties}
\label{sssec:signal_yield_syst}

Yield uncertainties act on the yield of a given reconstructed category, letting events migrate among them or changing the efficiency of the diphoton selection. 
Various experimental systematic sources relating to the photon selections are taken into account: photon uncertainties on \textit{isolation} and \textit{identification} efficiency, uncertainties on the efficiency of the diphoton \textit{trigger} and from \textit{pileup} modelling in simulation. 
From additional objects present in all the categories, the main experimental systematic sources of uncertainty are jet reconstruction uncertainties, especially jet flavour composition, flavour response, modelling, topology and jet energy resolution.
From per-event computed quantities, the dominant sources are the pileup reweighing (PRW) and the luminosity measurement. Both these uncertainties are fully correlated between all bins in all fiducial regions. 
The values for each category \emph{c} are computed as the relative difference between $\pm1\sigma$ varied yields and the nominal one using signal MC samples: 
\begin{equation}
\delta n_{c}^{\pm 1 \sigma} = \frac{n_{c}^{\pm 1 \sigma}}{n_{c}^\text{nom}} - 1
\label{eq:yield_syst}
\end{equation}
Each source of systematic uncertainty is treated as fully correlated within the categories. The nuisance parameters corresponding to these systematic uncertainties are entered into the fit with asymmetric constraints, as the up- and down-variations can  take different values. 
\textcolor{blue}{Plots to be included using h033}.






