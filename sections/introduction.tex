\section{Introduction}
\label{sec:introduction}
This note presents the details of the measurement of total and differential cross-sections of Higgs Boson decays to two photons ($H \to \gamma \gamma$) in $XXX \,\mathrm{fb}^{-1}$ of proton-proton collision data, collected at a center-of-mass energy of $13.6\,\mathrm{TeV}$ with the ATLAS detector at the Large Hadron Collider (LHC). The measurement follows previous analyses by ATLAS at $8 \,\mathrm{TeV}$ [CITATION], $13 \,\mathrm{TeV}$ [CITATION] and $13.6 \,\mathrm{TeV}$ in combination with $H \to ZZ$ [CITATION]. This analysis follows closely the strategy employed in the early Run 3 ATLAS measurement in $XXX \mathrm{fb}^{-1}$ at $13.6\,\mathrm{TeV}$, as well as the Run 2 analysis in $140 \,\mathrm{fb}^{-1}$ at $13\,\mathrm{TeV}$. Significant changes include updated CP recommendations, as well as modifications to set of variables studied (???) and the binning used in the histograms (???).

This section introduces the measurement, motivates its significance, and presents an overview of the analysis strategy. 

\subsection{Motivation}

During Run 3 of the LHC, ATLAS collected its largest dataset to-date at a record energy of $13.6\,\mathrm{TeV}$. This dataset provides a unique opportunity for precision measurement of the properties of the Higgs boson, including in the exceptionally well-resolved $H \to \gamma \gamma$ decay channel.

%%% TODO: A detailed motivation of the measurement should go here
% - The Run 2 INT note should be a good template https://cds.cern.ch/record/2714980/files/ATL-COM-PHYS-2020-253.pdf 
% - State the purpose of the measurement and provide some theoretical context 
% - Give a very brief description of the measurement
% - Describe fiducial sub-regions
% - Give a description of the kinematic variables for which we have studied the differential XSections

\subsection{Strategy}
% TODO: Describe the analysis strategy: event selelctions, binning, fitting, unfolding ...  
