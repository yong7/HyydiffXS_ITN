\section{Introduction}
\label{sec:introduction}
This note presents the measurement of the total and differential cross-sections
of Higgs Boson decays to two photons ($H \to \gamma \gamma$) in $168
\,\ifb$ of proton-proton collision data, collected at a
center-of-mass energy of $13.6\,\TeV$ with the ATLAS detector at the
Large Hadron Collider (LHC). The measurement follows previous analyses by ATLAS
at $8 \,\TeV$~\cite{HIGG-2013-10}, $13
\,\TeV$~\cite{HIGG-2022-04} and $13.6 \,\TeV$ in combination
with $H \to ZZ$~\cite{HIGG-2022-12}. This analysis follows closely the strategy
employed in the early Run 3 ATLAS measurements using $31.4\,\ifb$
at $13.6\,\TeV$, and Run 2 analysis using $140 \,\ifb$ at
$13\,\TeV$. As in the Run 2 analysis, the cross-section is measured in five phase-space regions:
the inclusive diphoton region dominated by the ggF production, as well as regions enhanced in VBF,\ \ZH, \WH and \ttH production (defined in Section \ref{sec:subregions_def}). In the diphoton region, differential cross-sections of eight variables ($\ptgg$, $\ygg$, $\Njets$, $\ptj$, $\dphijj$, $\mjj$, $\Nbjets$ and $\Nlept$) are measured. Significant changes made in this analysis include
updated CP recommendations and modifications to the set of variables studied.
A potential further change under study is a switch to so-called product cuts for the selection of diphoton events. 

This section introduces the measurement, motivates its significance, and presents an overview of the analysis strategy. 

\subsection{Motivation}

During LHC Run 3, the ATLAS experiment has recorded its largest dataset so far and at an increased proton-proton centre-of-mass energy of $13.6\,\mathrm{TeV}$. 
This dataset enables precision measurements of the Higgs boson properties including in the $H \to \gamma \gamma$ decay channel, where the signal appears as a narrow peak in the diphoton invariant mass spectrum.

A substantial background to $H \to \gamma \gamma$  production arises from non-resonant production of two high-energy photons that are not from Higgs boson decays. 
The normalisation and shape of the background is difficult to predict reliably from simulation, although it is not expected to peak at the Higgs boson mass. 
The analysis presented in this note therefore employs a data-driven technique to model the background.

A fiducial cross-section measures the production rate within a well-defined region of phase space, typically one that can be specified consistently at particle and detector level, and can therefore be used to compare observations and theoretical predictions. 
In this note, we define the \textit{inclusive fiducial volume}, a volume of phase space similar to the acceptance of the ATLAS detector.
The inclusive fiducial volume is defined as the region with two isolated photons with $|\eta|<1.37$ or $1.52<|\eta|<2.37$, and transverse momentum thresholds on the two photons of $p_{\text{T}}^{\gamma}/\myy> 0.35$ and $p_{\text{T}}^{\gamma}/\myy> 0.25$ where $\myy{}$ is the invariant mass of the diphoton system. 

Differential cross-section measurements examine how Higgs boson production varies across phase space, \textit{e.g.}~as a function of different kinematic variables and/or with additional particles in the final state.   
These distributions are sensitive to the properties of the Higgs boson, including its mass and couplings.  
%The Higgs boson is the most recently discovered fundamental particle, its properties, including couplings, are still not fully characterised.  
If the Higgs boson couples to beyond-the-Standard-Model particles this could become apparent in the tails of fiducial cross-section measurements.

By comparing the rate and shape of the differential cross-sections with theoretical predictions, we can look for deviations.
Any deviations may indicate new physics, or they could indicate limitations, or poor choice of parameters, in models we are comparing to.
In this note, we measure at the fiducial cross-section for $pp \to H \to \gamma \gamma$ production using variables that are shown to be sensitive to different production mechanisms and couplings.


\subsection{Differential Variables}

The differential cross-section is measured using several variables: 
the choice of which variables to use for the differential cross-section are motivated by the predicted significance of the distributions, from considerations of the different production mechanisms and with an aim to characterise the production as fully as possible given the data set.

The variables are described in detail in section~\ref{sec:observablesbinning}. They include kinematic properties of the photon-pair, the numbers of jets and charged leptons produced in association, and kinematic properties of any jet pairs.  The binning of the variables is chosen to optimise significance, to minimisation corrections applied in unfolding and to facilitate combinations with other analyses. 


\subsection{Analysis Strategy}
The strategy used in this analysis is similar to previous analyses~\cite{HIGG-2013-10, HIGG-2022-04, HIGG-2022-12} and is summarised briefly here:
\begin{itemize}
\item The data and simulated samples used are presented in Section~\ref{sec:samples};

\item Events are selected for further analysis, the selection criteria follow closely the definition of the fiducial region; 
the definition of the physics objects and event selection criteria is described in Section~\ref{sec:EventSelections}.

\item Differential cross-sections are defined by dividing the inclusive cross-section volume into several exclusive bins. The choice of variables and binning is presented in Section~\ref{sec:observablesbinning};

\item Section~\ref{sec:signalbackgroundmodelling} discusses the modelling of the background and signal as a function of the diphoton invariant mass $\mgg{}$;
analytical fits are made to understand the background in each fiducial cross-section bin and in the inclusive fiducial cross-section.

\item The residual background, including potential spurious signal is estimated. The main sources of background are non-resonant production of photon pairs and events where hadronic jets are misidentified as photons.  To estimate the background  the  \twoxtwod\ method is used; double two-dimensional sidebands are defined using $\mgg$ and changing the selection criteria on the photons. This is presented in Section~\ref{sec:backgroundestimation}.

\item Uncertainties due to systematic effects and theoretical modelling are explored and quantified in Section~\ref{sec:uncertainties}.

\item The fiducial cross-sections in terms of observed variables is highly dependent on the experimental setup (detector effects); in order to present results independent of the experiment, the cross-sections are \textit{unfolded} to remove detector effects. The unfolding procedure is presented in Section~\ref{sec:unfolding}.

\item The cross-section for the signal is extracted by fitting the observed $\mgg{}$ distribution to the signal and background as described in Section~\ref{sec:signalyieldextraction}.
  
\item Section~\ref{sec:asimovresults} presents the expected results; the observed results are presented in Sections~\ref{sec:dataresults}. Finally, conclusions are summarised in Section~\ref{sec:conclusion}.

\end{itemize}
