\section{Introduction}
\label{sec:introduction}
This note presents the measurement of the total and differential cross-sections
of Higgs Boson decays to two photons ($H \to \gamma \gamma$) in $168
\,\mathrm{fb}^{-1}$ of proton-proton collision data, collected at a
center-of-mass energy of $13.6\,\mathrm{TeV}$ with the ATLAS detector at the
Large Hadron Collider (LHC). The measurement follows previous analyses by ATLAS
at $8 \,\mathrm{TeV}$~\cite{HIGG-2013-10}, $13
\,\mathrm{TeV}$~\cite{HIGG-2022-04} and $13.6 \,\mathrm{TeV}$ in combination
with $H \to ZZ$~\cite{HIGG-2022-12}. This analysis follows closely the strategy
employed in the early Run 3 ATLAS measurements using $31.4\, \mathrm{fb}^{-1}$
at $13.6\,\mathrm{TeV}$, and Run 2 analysis using $140 \,\mathrm{fb}^{-1}$ at
$13\,\mathrm{TeV}$. Significant changes made in this analysis include using
updated CP recommendations and modifications to set of the studied variables.
Potential further change under consideration is a switch to to so-called
product di-photon cuts for the event selection. 

This section introduces the measurement, motivates its significance, and presents an overview of the analysis strategy. 

\subsection{Motivation}

During Run 3 of the LHC, the ATLAS experiment collected its largest dataset so far and at a new, higher, proton-proton collision energy of $13.6\,\mathrm{TeV}$. This dataset provides a unique opportunity for precision measurements of the Higgs boson’s properties, including in the $H \to \gamma \gamma$ decay channel, where the Higgs boson signal is well-resolved due to its narrow resonance mass.

However, there is a large background to the $\gamma\gamma$ final state due to production of two high-energy photons which are not produced from Higgs boson decay.  
The size and shape of this background is hard to predict, but should not peak around the Higgs mass.  The analysis presented in this note relies on a data-driven technique to model the background.

The fiducial cross section of a scattering process is a measurement of the cross section in a defined volume of phase space, typically one that can be easily and consistently defined in an experimental and theoretical context, and therefore can be used to make a comparison between observations and theoretical predictions. 
In this note, we define the \textit{inclusive fiducial volume} a volume of phase space similar to the acceptance of the ATLAS detector.

The inclusive fiducial volume is defined as the region to observe two isolated photons in the pseudorapidity range $|\eta|<1.37$ or $1.52<|\eta|<2.37$ and the photon transverse momentum satisfying $\ptgg > 0.35 (0.25)$; where $\ptgg{}$ is the invariant mass of the two photons. 

Differential cross section measurements looks at the how the production of the Higgs boson varies in different regions of phase space, e.g.~as a function of different kinematic variables with additional particles in the final state.   These distributions are sensitive to the properties of the Higgs boson, including its mass and couplings.  
%The Higgs boson is the most recently discovered fundamental particle, its properties, including couplings, are still not fully characterised.  
If the Higgs boson couples to BSM particles this might first become apparent in the tails of fiducial cross section measurements.

By comparing the rate and shape of the differential cross sections with theoretical predictions, we can look for any deviations; deviations could be a sign of new physics, or they could indicate a limitation or poor choice of parameters in the model we are comparing to.
In this note we look at the fiducial cross section for $pp \to H \to \gamma \gamma$ production using variables sensitive to different production mechanisms and couplings.

%% more detail to be added here about specific cuts 


\subsection{Differential Variables}

The differential cross section is measured using several different variables: 
the choice of which variables to use for the differential cross section are motivated by the predicted significance of the distributions, from considerations of the different production mechanisms and with an aim to characterise the production as fully as possible given the data set.

The variables are described in detail in section~\ref{sec:differentialobservables}.  They include kinematic properties of the photon-pair, the numbers of jets and charged leptons produced in association, and kinematic properties of any jet pairs.  The binning of the variables is chosen to optimise significance, to minimisation corrections apply at unfolding and to facilitate combination with other analyses. 


%%% TODO: A detailed motivation of the measurement should go here
% - The Run 2 INT note should be a good template https://cds.cern.ch/record/2714980/files/ATL-COM-PHYS-2020-253.pdf 
% - State the purpose of the measurement and provide some theoretical context 
% - Give a very brief description of the measurement
% - Describe fiducial sub-regions
% - Give a description of the kinematic variables for which we have studied the differential XSections

\subsection{Analysis Strategy}
The strategy used in this analysis is similar to previous analyses~\cite{HIGG-2013-10, HIGG-2022-04, HIGG-2022-12} and is summarised briefly here:
\begin{itemize}
\item The data and simulated samples used are presented in Section~\ref{sec:samples};

\item Events are selected for further analysis, the selection criteria follows closely the definition of the fiducial region; 
the definition of the physics objects and event selection criteria is described in Sections~\ref{sec:EventSelections} and~\ref{sec:fiducial}.

\item Differential cross sections are defined  by dividing the inclusive cross section volume into several exclusive bins.  The choice of variables and binning is presented in Section~\ref{sec:observablesbinning};

\item Section~\ref{sec:signalbackgroundmodelling} discusses the modelling of the background and signal as a function of the di-photon invariant mass $\mgg{}$;
analytical fits are made to understand the background in each fiducial cross section bin and in the inclusive fiducial cross section.



\item  The residual background, including potential spurious signal is estimated.  The main sources of background are non-resonant production of diphotons and events where hadronic jets are misreconstructed as photons.  To estimate the background  the  \twoxtwod\ method is used; double two-dimentional sidebands are defined using using $\mgg{}$ and changing the selection criteria used to select photons.  This is presented in Section~\ref{sec:backgroundestimation}.

\item  Uncertainties due to systematic effects and theoretical modelling are explored and quantified in Section~\ref{sec:uncertainties}.

\item The fiducial cross sections in terms of observed variables is highly
  dependent on the experimental setup (detector effects); in order to present
  results independent of the experiment, the cross sections are
  \textit{unfolded} to remove detector effects.  The unfolding procedure is
  presented in Section~\ref{sec:unfolding}.

\item The cross section for the signal is extracted by fitting the observed $\mgg{}$ distribution to the signal and background as described in Section~\ref{sec:signalyieldextraction}.
  
\item Section~\ref{sec:asimovresults} presents the expected results; the observed results and conclusions are presented in Sections~\ref{sec:dataresults} and~\ref{sec:conclusion}.

\end{itemize}
