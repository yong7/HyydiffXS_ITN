\section{Unfolding}
\label{sec:unfolding}

The fiducial differential cross-sections are corrected for detector effects using unfolding. The unfolding technique applied to obtain the nominal result is the matrix inversion method [or the bin-by-bin method], as was also used in the Run 2 measurement \cite{HIGG-2019-13} (described in detail in the corresponding INT note \cite{HIGG-2019-13-INT1}). 

Closure and effect of the unfolding procedure on systematic uncertainties are tested using the Asimov dataset. These studies are presented in Section \ref{sec:asimovresults}.

The following description of the unfolding procedure is taken from Ref \cite{HIGG-2019-13-INT1} with some minor edits for clarification.

The unfolding procedure accounts for the following effects:
\begin{itemize}
\item \textbf{resolution}: migration between bins at the detector and particle level;

\item \textbf{efficiency}: particle-level photons failing the detector level selection;

\item \textbf{fakes}: detector-level photons that do not have a particle level counterpart, either due to misreconstruction or migrations out of the fiducial volume. Diphoton Higgs decays where one of the photons is off-shell (so-called Dalitz decays) are also considered a source of fakes.
\end{itemize}

The signal cross-sections are extracted directly through the \mgg\ fit to the data.
The signal yields $\nu_i$ multiplying the signal \mgg\ PDF (in addition to the spurious
signal term), instead of themselves being the parameters of interest of the fit, are expressed in the likelihood function in terms of the cross-sections times BR, which are  then the parameters of interest of the fit. The parameterization is the following:
\begin{itemize}
\item in the bin-by-bin method:
  \begin{equation}
    \nu_i = \frac{C_i\sigma_i B_{\gamma\gamma} \cdot \mathcal{L}_\text{int}}{c_i^\textrm{Dalitz,OOF}};
  \end{equation}
\item in the response-matrix method:
  \begin{equation}
    \nu_i = \sum_{j}\frac{R_{ij} \sigma_j B_{\gamma\gamma} \cdot \mathcal{L}_\text{int}}{c_i^\textrm{Dalitz,OOF}}.
  \end{equation}
\end{itemize}
Here:
\begin{itemize}
\item $\mathcal{L}_\text{int}$ is the integrated luminosity,
\item $\sigma_i B_{\gamma\gamma}$ is the fiducial cross-section in bin $i$ times the Higgs boson branching ratio to diphotons,
\item $c_i^\textrm{Dalitz,OOF}$ is a correction factor determined from the simulation to subtract out-of-fiducial (OOF) and Dalitz events:
\begin{equation}
   c_i^\textrm{Dalitz,OOF} = \frac{n^\textrm{det}_i - n^\textrm{det,OOF}_i - n^\textrm{det,Dalitz}_i}{n^\textrm{det}_i},
\end{equation}
    {\em i.e.} the fitted signal yields $\nu_i$ are related to the yields $\nu_i^\textrm{sig}$
    of events within the fiducial volume by    
\begin{equation}
   \nu_i = \frac{\nu_i^\textrm{sig}}{c_i^\textrm{Dalitz,OOF}};
\end{equation}
\item 
$C_i$ is a correction factor determined as the ratio between the number of
fiducial events expected to pass the detector level criteria ($n_i^\text{det,fid}$)
and the number expected to pass the corresponding particle level criteria
($n_i^\text{ptcl}$):
\begin{equation}
  C_i = \frac{n_i^\text{det,fid}}{n_i^\text{ptcl}}
  \label{E. Correction factor}
\end{equation}
This takes into account both the efficiency of the selection and any
extrapolation between the detector and particle level phase space regions
considered.
\item
$R_{ij}$ is the response matrix that relates the number of
signal events reconstructed in a certain bin $j$ at detector level to the
number of events belonging to a certain bin $i$ at particle level:
$n_i^\text{det} = R_{ij}n_j^\text{ptcl}$.
Again, this includes both the efficiency of the selection and any
extrapolation between the detector and particle level phase space regions
considered.
\end{itemize}

The response matrices are obtained from the migration matrices (or yield
matrices) that tabulate in two dimensions the number of events in a given
bin $i$ of the particle level variable under study and in a given bin $j$
of the corresponding detector level quantity. 
% More details on the splitting of the response matrix into an efficiency term and a migration term are given in Appendix~\ref{App. response matrices details}, together with the plot of the corrections for fake candidates.

The migration matrices for the variables under study are shown in Section \ref{sec:observablesbinning}. The migration matrices, as well as the response matrices and correction factors
derived from them, are obtained by rescaling the matrices for each production mode
using the Standard Model cross-section, $\sigma_{s}^\text{SM}$ of each production
mode and the total initial weights generated, $N_{s}$.
For instance, the final correction factor applied to the data is
\begin{equation}
  C_i = \frac{\sum_s\sigma_s^\text{SM}\frac{n_{i,s}^\text{det,fid}}{N_{s}}}{\sum_s\sigma_s^\text{SM}\frac{n_{i,s}^\text{ptcl}}{N_{s}}}
\label{E. Correction factor combined}
\end{equation}
where $s\in\text{ggF},\text{VBF}, W^-H, W^+H, qq\to ZH, gg\to ZH, ttH, bbH, tWH, tHjb$.
% Dalitz events are excluded at both particle and detector level.

% ---- Text from Run 2 measuremnt INT note ----
% The measured, detector-level, signal yields are corrected for detector effects
% through an unfolding procedure that accounts for:
% \begin{itemize}
% \item resolution, leading to migration between bins of detector-level and
%   particle-level quantities;
% \item efficiencies, leading to a certain number of particle-level candidates
%   failing the detector-level selection;
% \item fakes, leading to a certain number of particle-level candidates that
%   do not belong to the fiducial volume but pass the detector-level selection.
%   The latter also corrects for Dalitz events.
% \end{itemize}

% Various unfolding procedures have been studied in past iterations of the
% analysis, with two methods being retained in the end:
% the bin-by-bin correction-factor, due to its small total uncertainties
% and limited bias, and the inversion of the response matrix (without
% regularisation) chosen as an alternative approach, which is unbiased but
% leads to larger statistical uncertainties in each bin (though they are
% correlated among bins and thus the total uncertainty is similar to the
% that of the bin-by-bin approach).

% The latest ATLAS $H\to 4\ell$ fiducial differential cross section measurement
% with the full Run2 data uses the matrix inversion approach as the baseline
% method, and for future combinations in a statistical code with that analysis
% (after extrapolating to the full volume) or with CMS we choose to present our
% results with both unfolding methods.
% To avoid the need of a regularisation procedure in the matrix inversion
% approach, the response matrix -- which gives the proability for an event,
% with true ({\em i.e.} particle-level) value of the variable under study
% in a certain bin $i$, to have a reconstructed (detector-level) value of
% the variable under study in a certain bin $j$ -- has to be well conditioned.
% This means that its \textbf{condition number}, defined as the ratio of the
% maximum and minimum singular values of the matrix, should be close to 1,
% leading to a low sensitivity to statistical fluctuations of the input.

% For this analysis, for consistency with the latest $H\to 4\ell$ publication, we
% decided to use the same unfolding approach, {\em i.e.} the one based on the
% response matrix. The bin-by-bin method was also used for comparison and
% as a backup option and is also described in the following, but the results
% that are shown in the following sections were obtained with the response matrix
% method.


% The signal cross sections are extracted directly through the \mgg fit to the data.
% The signal yields $\nu_i$ multiplying the signal \mgg pdf (in addition to the spurious
% signal term), instead of being the parameters of interest of the fit, are expressed
% in the likelihood function in terms of the cross sections times BR, which are then
% the parameters of interest of the fit. The parametrisation is the following:
% \begin{itemize}
% \item in the bin-by-bin method:
%   \begin{equation}
%     \nu_i = \frac{C_i\sigma_i B_{\gamma\gamma} \cdot \mathcal{L}_\text{int}}{c_i^\textrm{Dalitz,OOF}}
%   \end{equation}
% \item in the response-matrix method:
%   \begin{equation}
%     \nu_i = \sum_{j}\frac{R_{ij} \sigma_j B_{\gamma\gamma} \cdot \mathcal{L}_\text{int}}{c_i^\textrm{Dalitz,OOF}}
%   \end{equation}
% \end{itemize}
% Here:
% \begin{itemize}
% \item $\mathcal{L}_\text{int}$ is the integrated luminosity
% \item $\sigma_i B_{\gamma\gamma}$ is the fiducial cross-section in bin $i$ times the Higgs boson
%   branching ratio to diphotons.
% \item $c_i^\textrm{Dalitz,OOF}$ is a correction factor determined
%   from the simulation to subtract out-of-fiducial (OOF) and Dalitz
%   events:
% \begin{equation}
%    c_i^\textrm{Dalitz,OOF} = \frac{n^\textrm{det}_i - n^\textrm{det,OOF}_i - n^\textrm{det,Dalitz}_i}{n^\textrm{det}_i},
% \end{equation}
%     {\em i.e.} the fitted signal yields $\nu_i$ are related to the yields $\nu_i^\textrm{sig}$
%     of events within the fiducial volume by    
% \begin{equation}
%    \nu_i = \frac{\nu_i^\textrm{sig}}{c_i^\textrm{Dalitz,OOF}};
% \end{equation}
% \item 
% $C_i$ is a correction factor determined as the ratio between the number of
% fiducial events expected to pass the detector level criteria ($n_i^\text{det,fid}$)
% and the number expected to pass the corresponding particle level criteria
% ($n_i^\text{ptcl}$):
% \begin{equation}
%   C_i = \frac{n_i^\text{det,fid}}{n_i^\text{ptcl}}
%   \label{E. Correction factor}
% \end{equation}
% This takes into account both the efficiency of the selection and any
% extrapolation between the detector and particle level phase space regions
% considered.
% \item
% $R_{ij}$ is the response matrix that relates the number of
% signal events reconstructed in a certain bin $j$ at detector level to the
% number of events belonging to a certain bin $i$ at particle level:
% $n_i^\text{det} = R_{ij}n_j^\text{ptcl}$.
% Again, this includes both the efficiency of the selection and any
% extrapolation between the detector and particle level phase space regions
% considered.
% \end{itemize}

% The response matrices are obtained from the migration matrices (or yield
% matrices) that tabulate in two dimensions the number of events in a given
% bin $i$ of the particle-level variable under study and in a given bin $j$
% of the corresponding detector-level quantity. More details on the splitting
% of the response matrix into an efficiency term and a migration term are given
% in Appendix~\ref{App. response matrices details}, together with the plot of the corrections for fake candidates.

% The migration matrices for the variables under study are shown in
% Figures~\ref{fig:migration_matrices_inclusive_volumes}--\ref{fig:migration_matrices_VBF}.
% Dalitz events are excluded both at particle and detector level.
% The migration matrices and the response matrices and correction factors
% derived from them are obtained by rescaling the matrices for each production mode
% using the Standard Model cross section, $\sigma_{s}^\text{SM}$ of each production
% mode and the total initial weights generated, $N_{s}$.
% For instance, the final correction factor applied to the data is
% \begin{equation}
%   C_i = \frac{\sum_s\sigma_s^\text{SM}\frac{n_{i,s}^\text{det,fid}}{N_{s}}}{\sum_s\sigma_s^\text{SM}\frac{n_{i,s}^\text{ptcl}}{N_{s}}}
% \label{E. Correction factor combined}
% \end{equation}
% where $s\in\text{ggF},\text{VBF}, W^-H, W^+H, qq\to ZH, gg\to ZH, ttH, bbH, tWH, tHjb$.

% The response matrices are shown in Figures~\ref{fig:response_matrices_inclusive_volumes}--\ref{fig:response_matrices_VBF}.
% It should be noted that for the measurements in the inclusive fiducial regions (e.g. VBF) the
% response matrices also contain rows and columns for events outside that region (at particle or detector
% level) but are in the diphoton fiducial volume (or pass the diphoton inclusive selection) -- they will
% be called ``anti-VBF'', ``anti-ttH'' and so on in the following.
% In our fits we  measure simultaneously the event yield (or cross section) of events in the VBF
% and anti-VBF fiducial regions, where the latter provides a large sample of events that can help
% constraining the photon energy scale and resolution uncertainties in the fit.

% The condition numbers are summarised in Table~\ref{tab:unfolding_response_matrices_condition_numbers}.


% \subsection{Asimov dataset results}
% \label{subsec:unfoldasimov}
% \ASinote{Add unfolded results for the Run 3 Asimov dataset (Quratulain, Alex)}
