\newpage
%-------------------------------------------------------------------------------
\section{\File{atlasjournal.sty}}

Turn on including these definitions with the option \Option{journal=true} and off with the option \Option{journal=false}.

\input{atlasjournal}


\newpage
%-------------------------------------------------------------------------------
\section{\File{atlasmisc.sty}}

Turn on including these definitions with the option \Option{misc=true} and off with the option \Option{misc=false}.

\begin{xtabular}{ll}
\verb|\pT| & \pT \\
\verb|\pt| & \pt \\
\verb|\pTX| & \pTX \\
\verb|\ET| & \ET \\
\verb|\eT| & \eT \\
\verb|\et| & \et \\
\verb|\HT| & \HT \\
\verb|\pTsq| & \pTsq \\
\verb|\MET| & \MET \\
\verb|\met| & \met \\
\verb|\sumET| & \sumET \\
\verb|\EjetRec| & \EjetRec \\
\verb|\PjetRec| & \PjetRec \\
\verb|\EjetTru| & \EjetTru \\
\verb|\PjetTru| & \PjetTru \\
\verb|\EjetDM| & \EjetDM \\
\verb|\Rcone| & \Rcone \\
\verb|\abseta| & \abseta \\
\verb|\Ecm| & \Ecm \\
\verb|\rts| & \rts \\
\verb|\sqs| & \sqs \\
\verb|\Nevt| & \Nevt \\
\verb|\zvtx| & \zvtx \\
\verb|\dzero| & \dzero \\
\verb|\zzsth| & \zzsth \\
\verb|\RunOne| & \RunOne \\
\verb|\RunTwo| & \RunTwo \\
\verb|\RunThr| & \RunThr \\
\verb|\kt| & \kt \\
\verb|\antikt| & \antikt \\
\verb|\Antikt| & \Antikt \\
\verb|\pileup| & \pileup \\
\verb|\Pileup| & \Pileup \\
\verb|\btag| & \btag \\
\verb|\btagged| & \btagged \\
\verb|\bquark| & \bquark \\
\verb|\bquarks| & \bquarks \\
\verb|\bjet| & \bjet \\
\verb|\bjets| & \bjets \\
\verb|\mh| & \mh \\
\verb|\mW| & \mW \\
\verb|\mZ| & \mZ \\
\verb|\mH| & \mH \\
\verb|\ACERMC| & \ACERMC \\
\verb|\ALPGEN| & \ALPGEN \\
\verb|\AMCatNLO| & \AMCatNLO \\
\verb|\BLACKHAT| & \BLACKHAT \\
\verb|\CALCHEP| & \CALCHEP \\
\verb|\COLLIER| & \COLLIER \\
\verb|\COMPHEP| & \COMPHEP \\
\verb|\EVTGEN| & \EVTGEN \\
\verb|\FEYNRULES| & \FEYNRULES \\
\verb|\GGTOVV| & \GGTOVV \\
\verb|\GOSAM| & \GOSAM \\
\verb|\HATHOR| & \HATHOR \\
\verb|\Herwig| & \Herwig \\
\verb|\HERWIG| & \HERWIG \\
\verb|\HERWIGpp| & \HERWIGpp \\
\verb|\HRES| & \HRES \\
\verb|\JIMMY| & \JIMMY \\
\verb|\MADSPIN| & \MADSPIN \\
\verb|\MADGRAPH| & \MADGRAPH \\
\verb|\MGNLO| & \MGNLO \\
\verb|\MCatNLO| & \MCatNLO \\
\verb|\MCFM| & \MCFM \\
\verb|\METOP| & \METOP \\
\verb|\OPENLOOPS| & \OPENLOOPS \\
\verb|\POWHEG| & \POWHEG \\
\verb|\POWHEGBOX| & \POWHEGBOX \\
\verb|\POWHEGBOXRES| & \POWHEGBOXRES \\
\verb|\PHOTOS| & \PHOTOS \\
\verb|\PHOTOSpp| & \PHOTOSpp \\
\verb|\PROPHECY| & \PROPHECY \\
\verb|\PROTOS| & \PROTOS \\
\verb|\Pythia| & \Pythia \\
\verb|\PYTHIA| & \PYTHIA \\
\verb|\RECOLA| & \RECOLA \\
\verb|\Sherpa| & \Sherpa \\
\verb|\SHERPA| & \SHERPA \\
\verb|\TOPpp| & \TOPpp \\
\verb|\VBFNLO| & \VBFNLO \\
\verb|\MGNLOHER| & \MGNLOHER \\
\verb|\MGNLOPY| & \MGNLOPY \\
\verb|\MGHER| & \MGHER \\
\verb|\MGPY| & \MGPY \\
\verb|\POWHER| & \POWHER \\
\verb|\POWPY| & \POWPY \\
\verb|\SHERPABH| & \SHERPABH \\
\verb|\SHERPAOL| & \SHERPAOL \\
\verb|\ABM| & \ABM \\
\verb|\ABKM| & \ABKM \\
\verb|\CT| & \CT \\
\verb|\CTEQ| & \CTEQ \\
\verb|\GJR| & \GJR \\
\verb|\HERAPDF| & \HERAPDF \\
\verb|\LUXQED| & \LUXQED \\
\verb|\MSTW| & \MSTW \\
\verb|\MMHT| & \MMHT \\
\verb|\MSHT| & \MSHT \\
\verb|\NNPDF| & \NNPDF \\
\verb|\PDFforLHC| & \PDFforLHC \\
\verb|\AUET| & \AUET \\
\verb|\AZNLO| & \AZNLO \\
\verb|\FXFX| & \FXFX \\
\verb|\GEANT| & \GEANT \\
\verb|\MENLOPS| & \MENLOPS \\
\verb|\MEPSatLO| & \MEPSatLO \\
\verb|\MEPSatNLO| & \MEPSatNLO \\
\verb|\MINLO| & \MINLO \\
\verb|\Monash| & \Monash \\
\verb|\Perugia| & \Perugia \\
\verb|\Prospino| & \Prospino \\
\verb|\UEEE| & \UEEE \\
\verb|\LO| & \LO \\
\verb|\NLO| & \NLO \\
\verb|\NLL| & \NLL \\
\verb|\NNLO| & \NNLO \\
\verb|\muF| & \muF \\
\verb|\muQ| & \muQ \\
\verb|\muR| & \muR \\
\verb|\hdamp| & \hdamp \\
\verb|\NLOEWvirt| & \NLOEWvirt \\
\verb|\ra| & \ra \\
\verb|\la| & \la \\
\verb|\rarrow| & \rarrow \\
\verb|\larrow| & \larrow \\
\verb|\lapprox| & \lapprox \\
\verb|\rapprox| & \rapprox \\
\verb|\gam| & \gam \\
\verb|\stat| & \stat \\
\verb|\syst| & \syst \\
\verb|\radlength| & \radlength \\
\verb|\StoB| & \StoB \\
\verb|\dif| & \dif \\
\verb|\alphas| & \alphas \\
\verb|\NF| & \NF \\
\verb|\NC| & \NC \\
\verb|\CF| & \CF \\
\verb|\CA| & \CA \\
\verb|\TF| & \TF \\
\verb|\Lms| & \Lms \\
\verb|\Lmsfive| & \Lmsfive \\
\verb|\kperp| & \kperp \\
\verb|\Vcb| & \Vcb \\
\verb|\Vub| & \Vub \\
\verb|\Vtd| & \Vtd \\
\verb|\Vts| & \Vts \\
\verb|\Vtb| & \Vtb \\
\verb|\Vcs| & \Vcs \\
\verb|\Vud| & \Vud \\
\verb|\Vus| & \Vus \\
\verb|\Vcd| & \Vcd \\
\end{xtabular}


\noindent A length \Macro{figwidth} is defined that is \qty{2}{\cm} smaller than \Macro{textwidth}.

\noindent Monte Carlo generators and PDFs have an optional argument
that allows you to include the version, e.g.\
\verb|\PYTHIA[8]| to produce \PYTHIA[8] or\\
\verb|\PYTHIA[(v8.160)]| to produce \PYTHIA[(v8.160)].

\noindent A macro \Macro{pTX} is included that allows you to add an extra subscript and/or a superscript to \pT.
e.g.\\
\verb|\pTX| to produce \pTX;\\
\verb|\pTX[2]| to produce \pTX[2];\\
\verb|\pTX[][\text{had}]| to produce \pTX[][\text{had}];\\
\verb|\pTX[3][\text{had}]| to produce \pTX[3][\text{had}].

\noindent A generic macro \verb|\twomass| is defined, so that for example
\verb|\twomass{\mu}{\mu}| produces \twomass{\mu}{\mu} and \verb|\twomass{\mu}{e}| produces \twomass{\mu}{e}.

\noindent Several macros that help format numbers with rounding are provided:
e.g.\\
\verb|\numR{3.145159}{2}| \(\rightarrow\) \(\numR{3.14159}{2}\)\\
\verb|\numRF{3.145159}{3}| \(\rightarrow\) \(\numRF{3.14159}{3}\)\\
\verb|\numRP{3.145159}{3}| \(\rightarrow\) \(\numRP{3.14159}{3}\)\\
\verb|\numpmerr{+1.234}{-3.456}{2}| \(\rightarrow\) \(\numpmerr{+1.234}{-0.3456}{2}\)\\[0.5ex]
\verb|\numpmRF{+1.234}{-3.456}{2}|  \(\rightarrow\) \(\numpmRF{+1.234}{-0.3456}{2}\)\\[0.5ex]
\verb|\numpmRP{-1.234}{+3.456}{2}|  \(\rightarrow\) \(\numpmRP{-1.234}{+3.456}{2}\)\\[0.5ex]
\verb|\numpmerrx[2][figures]{+1.234}{-3.456}| \(\rightarrow\) \(\numpmerrx[2][figures]{+1.234}{-3.456}\)\\[0.5ex]
Note that \Macro{numR} and \Macro{numpmerr} only perform rounding if \Option{round-mode} is set to something
via \Macro{sisetup} (which is not the case here).
The same applies if the second argument of \Macro{numpmerrx} is omitted.
See the \Package{siunitx} package for more details.

The macro \Macro{supsub} ensures that superscripts and subscripts have the same width.
This is useful for asymmetric errors if the widths of \enquote{\(+\)} and \enquote{\(-\)}
are not the same.

A macro \verb|\dk| is defined which makes it easier to write down decay chains.
For example
\begin{verbatim}
\[\eqalign{a \to & b+c\\
   & \dk & e+f \\
   && \dk g+h}
\]
\end{verbatim}
produces
\[\eqalign{a \to & b+c\cr
   & \dk & e+f \cr
   && \dk g+h}
\]
Note that \Macro{eqalign} is also redefined in this package so that \Macro{dk} works.

The following macro names have been changed:\\
\verb|\ptsq| \(\to\) \verb|\pTsq|.


\newpage
%-------------------------------------------------------------------------------
\section{\File{atlasxref.sty}}

Turn on including these definitions with the option \Option{xref=true} and off with the option \Option{xref=false}.

\input{atlasxref}

\noindent The following macros with arguments are also defined:
\begin{xtabular}{ll}
\verb|\App{1}|  & \App{1}\\
\verb|\Eqn{1}|  & \Eqn{1}\\
\verb|\Fig{1}|  & \Fig{1}\\
\verb|\Refn{1}|  & \Refn{1}\\
\verb|\Sect{1}| & \Sect{1}\\
\verb|\Tab{1}|  & \Tab{1}\\
\verb|\Apps{1}{4}| & \Apps{1}{4} \\
\verb|\Eqns{1}{4}| & \Eqns{1}{4} \\
\verb|\Figs{1}{4}| & \Figs{1}{4} \\
\verb|\Refns{1}{4}| & \Refns{1}{4} \\
\verb|\Sects{1}{4}| & \Sects{1}{4} \\
\verb|\Tabs{1}{4}| & \Tabs{1}{4} \\
\verb|\Apprange{1}{4}| & \Apprange{1}{4} \\
\verb|\Eqnrange{1}{4}| & \Eqnrange{1}{4} \\
\verb|\Figrange{1}{4}| & \Figrange{1}{4} \\
\verb|\Refrange{1}{4}| & \Refrange{1}{4} \\
\verb|\Sectrange{1}{4}| & \Sectrange{1}{4} \\
\verb|\Tabrange{1}{4}| & \Tabrange{1}{4}
\end{xtabular}

The idea is that you can adapt these definitions according to your own preferences (or those of a journal).
Note that the macros \Macro{Ref} and \Macro{Refs} were renamed to \Macro{Refn} and \Macro{Refns}
in \Package{atlaslatex} 08-00-00, as \Macro{Ref} is now defined in the \Package{hyperref} package.


\newpage
%-------------------------------------------------------------------------------
\section{\File{atlasbsm.sty}}

Turn on including these definitions with the option \Option{BSM} and off with the option \Option{BSM=false}.

The macro \Macro{susy} simply puts a tilde (\(\tilde{\ }\)) over its argument,
e.g.\ \verb|\susy{q}| produces \susy{q}.

For \susy{q}, \susy{t}, \susy{b}, \slepton, \sel, \smu and
\stau, L and R states are defined; for stop, sbottom and stau also the
light (1) and heavy (2) states.
There are four neutralinos and two charginos defined,
the index number unfortunately needs to be written out completely.
For the charginos the last letter(s) indicate(s) the charge:
\enquote{p} for \(+\), \enquote{m} for \(-\), and \enquote{pm} for \(\pm\).

\input{atlasbsm}


\newpage
%-------------------------------------------------------------------------------
\section{\File{atlasheavyion.sty}}

Turn on including these definitions with the option \Option{hion=true} and off with the option \Option{hion=false}.
The heavy ion definitions use the package \Package{mhchem} to help with the formatting of chemical elements.
This package is included by \File{atlasheavyion.sty}.

\input{atlasheavyion}

%The following symbols were removed or modified with respect to the original submission
%\input{atlasheavyion-mod}


\newpage
%-------------------------------------------------------------------------------
\section{\File{atlasjetetmiss.sty}}

Turn on including these definitions with the option \Option{jetetmiss=true} and off with the option \Option{jetetmiss=false}.

\input{atlasjetetmiss}

\noindent The macro \Macro{etaRange} produces what you would expect:
\verb|\etaRange{-2.5}{+2.5}| produces \etaRange{-2.5}{+2.5} while
\verb|\AetaRange{1.0}| produces \AetaRange{1.0}.
The macro \Macro{avg} can be used for average values:
\verb|\avg{\mu}| produces \avg{\mu}.


% \newpage
%-------------------------------------------------------------------------------
% \section{\File{atlasmath.sty}}

% Turn on including these definitions with the option \Option{math=true} and off with the option \Option{math=false}.

% \input{atlasmath}

% \noindent The macro \Macro{spinor} is also defined.
% \verb|\spinor{u}| produces \spinor{u}.


% \newpage
%-------------------------------------------------------------------------------
% \section{\File{atlasother.sty}}

% Turn on including these definitions with the option \Option{other} and off with the option \Option{other=false}.

% \input{atlasother}
