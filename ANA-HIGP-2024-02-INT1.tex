%-------------------------------------------------------------------------------
% This file provides a skeleton ATLAS note.
\pdfinclusioncopyfonts=1
% This command may be needed in order to get \ell in PDF plots to appear. Found in
% https://tex.stackexchange.com/questions/322010/pdflatex-glyph-undefined-symbols-disappear-from-included-pdf
%-------------------------------------------------------------------------------
% Specify where ATLAS LaTeX style files can be found.
\RequirePackage{latex/atlaslatexpath}
% You can comment out the above line if the files are in a central location, e.g. $HOME/texmf.
%-------------------------------------------------------------------------------
\documentclass[NOTE, REPORT=true, atlasdraft=true, UKenglish]{atlasdoc}
% The language of the document must be set: usually UKenglish or USenglish.
% british and american also work!
% Commonly used options:
%  atlasdraft=true|false This document is an ATLAS draft.
%  texlive=YYYY          Specify TeX Live version (2020 is default).
%  NOTE                  The document is an ATLAS note (draft).
%  REPORT=true|false     Use scrreprt|scrartcl as main class for document.
%  txfonts=true|false    Use txfonts rather than the default newtx.
%  paper=a4|letter       Set paper size to A4 (default) or letter.

%-------------------------------------------------------------------------------
% Extra packages:
\usepackage{atlaspackage}
% Commonly used options:
%  backend=bibtex        Use the bibtex backend rather than biber.
%  subfigure|subfig|subcaption  to use one of these packages for figures in figures.
%  minimal               Minimal set of packages.
%  default               Standard set of packages.
%  full                  Full set of packages.
%-------------------------------------------------------------------------------
% Style file with biblatex options for ATLAS documents.
\usepackage{atlasbiblatex}
% Commonly used options:
%  backref=true|false    Turn on or off back references in the bibliography.

% Package for creating list of authors and contributors to the analysis.
\usepackage{atlascontribute}

% Useful macros
\usepackage{atlasphysics}
% See doc/atlas_physics.pdf for a list of the defined symbols.
% Default options are:
%   true:  journal, misc, particle, unit, xref
%   false: BSM, hepparticle, hepprocess, hion, jetetmiss, math, process,
%          other, snippets, texmf
% See the package for details on the options.

% Macro to add to-do notes (for several authors). Uses the todonotes package.
% \ATLnote{JS}{Jane}{green!20}{green!50!black!60}
% add macros \JSnote and \JSinote for notes in the margin and inline.
% The first colour is for the body and the second for the border of the note.
% Set output=false in order not to print out the notes.
% Set shift=false to avoid adjustment of margins.
% Check for notes still left by commenting out the package in the final version of the note.
\usepackage[output=true, shift=true]{atlastodo}

% Files with references for use with biblatex.
% Note that biber gives an error if it finds empty bib files.
% \addbibresource{ANA-HIGP-2024-02-INT1.bib}
\addbibresource{bib/ATLAS.bib}
\addbibresource{bib/CMS.bib}
\addbibresource{bib/ConfNotes.bib}
\addbibresource{bib/PubNotes.bib}

% Paths for figures - do not forget the / at the end of the directory name.
\graphicspath{{logos/}{figures/}}

% Add your own definitions here (file ANA-HIGP-2024-02-INT1-defs.sty).
\usepackage{ANA-HIGP-2024-02-INT1-defs}

%-------------------------------------------------------------------------------
% Generic document information.
%-------------------------------------------------------------------------------

% Title, abstract and document.
%-------------------------------------------------------------------------------
% This file contains the title, author and abstract.
% It also contains all relevant document numbers used for an ATLAS note.
%-------------------------------------------------------------------------------

% Title
\AtlasTitle{Measurement of fiducial and differential cross sections in the $H\rightarrow\gamma\gamma$ decay channel with 168 $\text{fb}^{-1}$ of 13.6 $\text{TeV}$ proton–proton collision data with the ATLAS detector}

% Draft version:
% Should be 1.0 for the first circulation, and 2.0 for the second circulation.
% If given, adds draft version on front page, a 'DRAFT' box on top of each other page, 
% and line numbers.
% Comment or remove in final version.
\AtlasVersion{0.1}

% Abstract - % directly after { is important for correct indentation
\AtlasAbstract{%
  We present supporting documentation for fiducial and differential cross-section measurements of the Higgs boson in the $H\rightarrow\gamma\gamma$ decay channel using 168 $\text{fb}^{-1}$of proton--proton collisions recorded at 13.6 $\text{TeV}$ centre of-mass energy in the years 2022 to 2024.
  %The amount of background, mainly from SM diphoton production and hadronic jets, is determined by a simultaneous signal and background fit to the diphoton mass spectrum.
  The fiducial cross-section is measured in few inclusive phase space regions, and differentially as functions of a selection of variables.
  %The data is compared to several state-of-the-art theoretical predictions of SM Higgs boson production.
}

% Author - this does not work with revtex (add it after \begin{document})
\author{The ATLAS Collaboration}

% Authors and list of contributors to the analysis
% \AtlasAuthorContributor also adds the name to the author list
% Include package latex/atlascontribute to use this
% Use authblk package if there are multiple authors, which is included by latex/atlascontribute
% \usepackage{authblk}
% Use the following 3 lines to have all institutes on one line
% \makeatletter
% \renewcommand\AB@affilsepx{, \protect\Affilfont}
% \makeatother
% \renewcommand\Authands{, } % avoid ``. and'' for last author
% \renewcommand\Affilfont{\itshape\small} % affiliation formatting
\AtlasAuthorContributor{HUSTON, Joey}{a}{Product-cut studies}
\AtlasAuthorContributor{JIN, Shan}{b}{ }
\AtlasAuthorContributor{MIJOVIC, Liza}{c}{Analysis contact, Signal modelling, Asimov fits}
\AtlasAuthorContributor{MARTIN, Victoria}{c}{Supervisor/PI of Julia Allen, Alex Sopio.}
\AtlasAuthorContributor{TURRA, Ruggero}{d}{ }
\AtlasAuthorContributor{MANZONI, Stefano}{e}{ }
\AtlasAuthorContributor{CUETO GOMEZ, Ana Rosario}{f}{Variable selection \& binning optimisation.}
\AtlasAuthorContributor{TAREK, Ahmed}{e}{ }
\AtlasAuthorContributor{D'ERAMO, Louis}{h}{ }
\AtlasAuthorContributor{MUNGO, Davide}{i}{ }
\AtlasAuthorContributor{XIA, Ligang}{b}{ }
\AtlasAuthorContributor{PASCUAL DOMINGUEZ, Luis}{f}{Analysis contact}
\AtlasAuthorContributor{SOPIO, Alex}{c}{INT Note editor, 2x2D background estimation.}
\AtlasAuthorContributor{NASELLA, Laura}{d}{ }
\AtlasAuthorContributor{KAFLE, Pratik}{a}{Product-cut studies}
\AtlasAuthorContributor{ALLEN, Julia Frances}{c}{Signal modelling, mc21 vs. mc23 cross-checks.}
\AtlasAuthorContributor{ZHOU, Yong}{b}{INT Note editor, spurious signal.}
\AtlasAuthorContributor{COLLADO SOTO, Pablo}{f}{ }

%\AtlasContributor{Fourth AtlasContributor}{b}{Contribution to the analysis.}
%\author[a]{First Author}
% \author[a]{Second Author}
% \author[b]{Third Author}
\affil[a]{Michigan State University}
\affil[b]{Nanjing University}
\affil[c]{University of Edinburgh}
\affil[d]{Università degli Studi di Milano}
\affil[e]{CERN}
\affil[f]{Universidad de Madrid}
\affil[h]{Université Clermont Auvergne}
\affil[i]{University of Toronto}


% If a special author list should be indicated via a link use the following code:
% Include the two lines below if you do not use atlasstyle:
% \usepackage[marginal,hang]{footmisc}
% \setlength{\footnotemargin}{0.5em}
% Use the following lines in all cases:
% \usepackage{authblk}
% \author{The ATLAS Collaboration%
% \thanks{The full author list can be found at:\newline
%   \url{https://atlas.web.cern.ch/Atlas/PUBNOTES/ATL-PHYS-PUB-2016-007/authorlist.pdf}}
% }

% ATLAS reference code, to help ATLAS members to locate the paper
\AtlasRefCode{ANA-HIGP-2024-02}

% ATLAS note number. Can be an COM, INT, PUB or CONF note
\AtlasNote{ANA-HIGP-2024-02-INT1}

% Author and title for the PDF file.
\hypersetup{pdftitle={ATLAS document},pdfauthor={The ATLAS Collaboration}}

%-------------------------------------------------------------------------------
% Content
%-------------------------------------------------------------------------------
\begin{document}

\maketitle

% List of contributors - print here or after the Bibliography.
% \PrintAtlasContribute{0.30}
% \clearpage

% List of to-do notes.
% \listoftodos

\section*{Changelog}

\begin{itemize}
    \item v0.1: First version of the document.
    \begin{itemize}
        \item Added list of contributions, publication timeline and to-dos to front matter.
    \end{itemize}
\end{itemize}

\section*{List of contributions}

\begin{table}[h]
    \centering
    \begin{tabular}{|l|p{7cm}|}
        \hline
        \textbf{Author} & \textbf{Contribution}  \\
        \hline
        HUSTON, Joey (Michigan SU)	                &  \\
        JIN, Shan (Nanjing)	                        &  \\
        MIJOVIC, Liza (Edinburgh)	                &  \\
        MARTIN, Victoria (Edinburgh)	            & PI/Supervisor of Julia Allen, Alex Sopio \\
        TURRA, Ruggero (Milano)	                    &  \\
        MANZONI, Stefano (CERN)	                    &  \\
        CUETO GOMEZ, Ana Rosario (Madrid UA)	    & Variable selection \& binning optimisation \\
        TAREK, Ahmed (CERN)	                        &  \\
        D'ERAMO, Louis (Clermont-Ferrand)	        &  \\
        MUNGO, Davide (Toronto)	                    &  \\
        XIA, Ligang (Nanjing)	                    &  \\
        PASCUAL DOMINGUEZ, Luis (Madrid UA)	        &  \\
        SOPIO, Alex (Edinburgh)	                    & INT Note editor, 2x2D background estimation \\
        NASELLA, Laura (Milano)	                    &  \\
        KAFLE, Pratik (Michigan SU)	                &  \\
        ALLEN, Julia Frances (Edinburgh)	        & Signal fits, mc21 vs. mc23 cross-checks \\
        ZHOU, Yong (Nanjing)	                    & INT Note editor, spurious signal \\
        COLLADO SOTO, Pablo (Madrid UA)	            &  \\
        \hline
    \end{tabular}
    \caption{List of contributions to the analysis.}
    \label{tab:contributions}
\end{table}


\section*{Publication timeline}

\begin{table}[h]
    \centering
    \begin{tabular}{|l|l|}
        \hline
        \textbf{Date} & \textbf{Event}  \\
        \hline
        2025-08 & EB request \\
        2025-11 & Higgs 2025 \\
        \hline
    \end{tabular}
    \caption{Analysis timeline as of XX of XX 2025.}
\end{table}

\section*{To-do list}

\begin{itemize}
    \item Things to do go here.
\end{itemize}



\tableofcontents

%-------------------------------------------------------------------------------
\chapter{Introduction}
\label{ch:intro}
%-------------------------------------------------------------------------------
\section{Introduction}
\label{sec:introduction}
This note presents the measurement of the total and differential cross-sections
of Higgs Boson decays to two photons ($H \to \gamma \gamma$) in $168
\,\mathrm{fb}^{-1}$ of proton-proton collision data, collected at a
center-of-mass energy of $13.6\,\mathrm{TeV}$ with the ATLAS detector at the
Large Hadron Collider (LHC). The measurement follows previous analyses by ATLAS
at $8 \,\mathrm{TeV}$~\cite{HIGG-2013-10}, $13
\,\mathrm{TeV}$~\cite{HIGG-2022-04} and $13.6 \,\mathrm{TeV}$ in combination
with $H \to ZZ$~\cite{HIGG-2022-12}. This analysis follows closely the strategy
employed in the early Run 3 ATLAS measurements using $31.4\, \mathrm{fb}^{-1}$
at $13.6\,\mathrm{TeV}$, and Run 2 analysis using $140 \,\mathrm{fb}^{-1}$ at
$13\,\mathrm{TeV}$. Significant changes made in this analysis include using
updated CP recommendations and modifications to set of the studied variables.
Potential further change under consideration is a switch to to so-called
product di-photon cuts for the event selection. 

This section introduces the measurement, motivates its significance, and presents an overview of the analysis strategy. 

\subsection{Motivation}

During Run 3 of the LHC, the ATLAS experiment collected its largest dataset so far and at a new, higher, proton-proton collision energy of $13.6\,\mathrm{TeV}$. This dataset provides a unique opportunity for precision measurements of the Higgs boson’s properties, including in the $H \to \gamma \gamma$ decay channel, where the Higgs boson signal is well-resolved due to its narrow resonance mass.

However, there is a large background to the $\gamma\gamma$ final state due to production of two high-energy photons which are not produced from Higgs boson decay.  
The size and shape of this background is hard to predict, but should not peak around the Higgs mass.  The analysis presented in this note relies on a data-driven technique to model the background.

The fiducial cross section of a scattering process is a measurement of the cross section in a defined volume of phase space, typically one that can be easily and consistently defined in an experimental and theoretical context, and therefore can be used to make a comparison between observations and theoretical predictions. 
In this note, we define the \textit{inclusive fiducial volume} a volume of phase space similar to the acceptance of the ATLAS detector.

The inclusive fiducial volume is defined as the region to observe two isolated photons in the pseudorapidity range $|\eta|<1.37$ or $1.52<|\eta|<2.37$ and the photon transverse momentum satisfying $\ptgg > 0.35 (0.25)$; where $\ptgg{}$ is the invariant mass of the two photons. 

Differential cross section measurements looks at the how the production of the Higgs boson varies in different regions of phase space, e.g.~as a function of different kinematic variables with additional particles in the final state.   These distributions are sensitive to the properties of the Higgs boson, including its mass and couplings.  
%The Higgs boson is the most recently discovered fundamental particle, its properties, including couplings, are still not fully characterised.  
If the Higgs boson couples to BSM particles this might first become apparent in the tails of fiducial cross section measurements.

By comparing the rate and shape of the differential cross sections with theoretical predictions, we can look for any deviations; deviations could be a sign of new physics, or they could indicate a limitation or poor choice of parameters in the model we are comparing to.
In this note we look at the fiducial cross section for $pp \to H \to \gamma \gamma$ production using variables sensitive to different production mechanisms and couplings.

%% more detail to be added here about specific cuts 


\subsection{Differential Variables}

The differential cross section is measured using several different variables: 
the choice of which variables to use for the differential cross section are motivated by the predicted significance of the distributions, from considerations of the different production mechanisms and with an aim to characterise the production as fully as possible given the data set.

The variables are described in detail in section~\ref{sec:differentialobservables}.  They include kinematic properties of the photon-pair, the numbers of jets and charged leptons produced in association, and kinematic properties of any jet pairs.  The binning of the variables is chosen to optimise significance, to minimisation corrections apply at unfolding and to facilitate combination with other analyses. 


%%% TODO: A detailed motivation of the measurement should go here
% - The Run 2 INT note should be a good template https://cds.cern.ch/record/2714980/files/ATL-COM-PHYS-2020-253.pdf 
% - State the purpose of the measurement and provide some theoretical context 
% - Give a very brief description of the measurement
% - Describe fiducial sub-regions
% - Give a description of the kinematic variables for which we have studied the differential XSections

\subsection{Analysis Strategy}
The strategy used in this analysis is similar to previous analyses~\cite{HIGG-2013-10, HIGG-2022-04, HIGG-2022-12} and is summarised briefly here:
\begin{itemize}
\item The data and simulated samples used are presented in Section~\ref{sec:samples};

\item Events are selected for further analysis, the selection criteria follows closely the definition of the fiducial region; 
the definition of the physics objects and event selection criteria is described in Sections~\ref{sec:EventSelections} and~\ref{sec:fiducial}.

\item Differential cross sections are defined  by dividing the inclusive cross section volume into several exclusive bins.  The choice of variables and binning is presented in Section~\ref{sec:observablesbinning};

\item Section~\ref{sec:signalbackgroundmodelling} discusses the modelling of the background and signal as a function of the di-photon invariant mass $\mgg{}$;
analytical fits are made to understand the background in each fiducial cross section bin and in the inclusive fiducial cross section.



\item  The residual background, including potential spurious signal is estimated.  The main sources of background are non-resonant production of diphotons and events where hadronic jets are misreconstructed as photons.  To estimate the background  the  \twoxtwod\ method is used; double two-dimentional sidebands are defined using using $\mgg{}$ and changing the selection criteria used to select photons.  This is presented in Section~\ref{sec:backgroundestimation}.

\item  Uncertainties due to systematic effects and theoretical modelling are explored and quantified in Section~\ref{sec:uncertainties}.

\item The fiducial cross sections in terms of observed variables is highly
  dependent on the experimental setup (detector effects); in order to present
  results independent of the experiment, the cross sections are
  \textit{unfolded} to remove detector effects.  The unfolding procedure is
  presented in Section~\ref{sec:unfolding}.

\item The cross section for the signal is extracted by fitting the observed $\mgg{}$ distribution to the signal and background as described in Section~\ref{sec:signalyieldextraction}.
  
\item Section~\ref{sec:asimovresults} presents the expected results; the observed results and conclusions are presented in Sections~\ref{sec:dataresults} and~\ref{sec:conclusion}.

\end{itemize}

\section{Data and Simulated Samples}
\label{sec:samples}

%-------------------------------------------------------------------------------
\chapter{Analysis}
\label{ch:analysis}
%-------------------------------------------------------------------------------

You can find some text snippets that can be used in papers in \texttt{latex/atlassnippets.sty}.
To use them, provide the \texttt{snippets} option to \texttt{atlasphysics}.

%-------------------------------------------------------------------------------
\chapter{Results}
\label{ch:result}
%-------------------------------------------------------------------------------

Place your results here.

% All figures and tables should appear before the summary and conclusion.
% The package placeins provides the macro \FloatBarrier to achieve this.
% \FloatBarrier

%-------------------------------------------------------------------------------
\chapter{Conclusion}
\label{ch:conclusion}
%-------------------------------------------------------------------------------

Place your conclusion here.

%-------------------------------------------------------------------------------
% If you use biblatex and either biber or bibtex to process the bibliography
% just say \printbibliography here.
\printbibliography
% If you want to use the traditional BibTeX you need to use the syntax below.
% \bibliographystyle{obsolete/bst/atlasBibStyleWithTitle}
% \bibliography{ANA-HIGP-2024-02-INT1,bib/ATLAS,bib/CMS,bib/ConfNotes,bib/PubNotes}
%-------------------------------------------------------------------------------

%-------------------------------------------------------------------------------
% Print the list of contributors to the analysis.
% The argument gives the fraction of the text width used for the names.
%-------------------------------------------------------------------------------
\clearpage
The supporting notes for the analysis should also contain a list of contributors.
This information should usually be included in \texttt{mydocument-metadata.tex}.
The list should be printed either here or before the Table of Contents.
\PrintAtlasContribute{0.30}

%-------------------------------------------------------------------------------
\clearpage
\appendix
\part*{Appendices}
\addcontentsline{toc}{part}{Appendices}
%-------------------------------------------------------------------------------

In an ATLAS note, use the appendices to include all the technical details of your work
that are relevant for the ATLAS Collaboration only (e.g.\ dataset details, software release used).
This information should be printed after the Bibliography.

\end{document}
