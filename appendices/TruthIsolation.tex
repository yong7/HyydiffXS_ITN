\section{Truth Isolation Studies}
\label{sec:truthiso}

Dedicated studies on a photon isolation selection at truth level that well matches with the reconstruction efficiencies were studied in detail in ~\cite{ATL-COM-PHYS-2018-202}.
Similar studies have been conducted with Run 3 running conditions with the aim of cross checking that the same conclusions reached in that case, still hold.

\subsection{Mapping detector level isolation cut}
The isolation requirement at isolation level is mapped to that at truth-level. In the case of the track (calorimeter) isolation, the following recipe is used:

\begin{itemize}

\item \textit{Detector level:} Apply the full cutflow including only calorimeter (track) isolation rejecting events without a reconstructed diphoton system.
\item \textit{Particle level:} Select stable photons that are not originating from hadrons with $p_{T} > 20$~GeV.
\item Match detector level and particle level leading photon cadidate with $\Delta R < 0.1$
\item Fill a 2D histogram with detector and particle level track-isolation (\textit{etcone}) computed as described in Section~\ref{sec:EventSelections}. In case of \textit{etcone}, the particle-level isolation is computed for the transverse momentum of the sum of all particles momenum around the photon in a radius of $\Delta R=0.2$ around the photon excluding muons and neutrinos.
\item Profile the histogram using the median to account for the peak at zero coming from true isolated photons in the track isolation. The error is computed using quantiles as an approximation.
\end{itemize}  

The results of this mapping are shown in Figure~\ref{fig:truthisomap}. The relation between the particle-level and detector-level isolation is defined as:

$$ ptcone20_{detector} = p_0 +p_1 \times ptcone20_{particle}$$

$$ topoetcone20_{detector} = p_0 +p_1 \times etcone20_{particle}$$

The resulting mapping fit for track isolation is:
$$ p_1 = 1.001 \pm 0.00345,\quad p_0 = -0.057 \pm 0.03 GeV $$

showing that the variables are equivalent. And, thus, the same cut applied at detector leve, can be applied at particle level $ptcone20_{particle}<0.05 \times p_{T}$. The case of calorimetric isolation is different with a non-linear response of the isolation values. A linear fit provides a fitting values of:
$$ p_1 = 0.339 \pm 0.027,\quad p_0 = -0.462 \pm 0.212 GeV $$

meaning that to get a particle-level \textit{etcone} requirement equivalent to the detector level one, we should apply a cut of $ etcone20_{particle} < 0.192 p_{T} +1.365$~GeV.


\begin{figure}[h]
  \begin{center}
    \includegraphics[width=0.45\textwidth]{figures/TruthIso/PtConeFit.pdf}
    \includegraphics[width=0.45\textwidth]{figures/TruthIso/EtConeFit.pdf}
    \caption{Profile of the 2D histogram with the linear fit to determine detector level isolation as function of the particle level isolation for charged-particles isolation and \textit{etcone}}
    \label{fig:truthisomap}
    % \label{fig:efficiency_binning} %probably wrong label
  \end{center}
\end{figure}



It was shown in ~\cite{ATL-COM-PHYS-2018-202}, that track isolation is by-far the most restrictive isolation cut in the fiducial volume, while \textit{etcone} only has a negligible effect. These conclusions were cross-checked using the  \textit{etcone} isolation requirement at particle level as obtained above. The impact on the unfolding correction factors are shown in Figure~\ref{fig:corrfaciso}. For comparison, the correction factors obtained with an analogous cut obtained in the Run 2 studies are also shown. The results show that:

\begin{itemize}

\item isolation corrects for model-dependence seen in the case in which isolation is not applied with correction factors more homogeneous across the different production modes;
\item the restriction of the phase imposed by the \textit{etcone20} isolation is neglibile on top of the \textit{ptcone20} isolation;
\item while a linear fit in the mapping of the calorimeter isolation is not ideal, changes in those values do not change the previous conclusion (as seen by the practical absence of diffences between the obtained Run 2 and  Run 3 \textit{etcone20} cut);
\item a particle-level isolation requirement on charged particles using the same cuts as at detector level can be applied to mimic the detector-level isolation.  
  
\end{itemize}    

    
\begin{figure}[h]
  \begin{center}
    \includegraphics[width=0.8\textwidth]{figures/TruthIso/CorrectionFactorsRuns23.pdf}
    
    \caption{Unfolding corrections factors obtained fro the different production modes without particle-level isolation, applying only \textit{ptcone20} at particle level or applying \textit{ptcone20} and \textit{etcone20} isolation at particle level}
    \label{fig:corrfaciso}
  \end{center}
\end{figure}


