\section{Product Cut Studies}
\label{app:productcut}
Higgs boson production (through gluon-gluon fusion) has been calculated at $N^3LO$~\cite{Anastasiou:2015vya,Anastasiou:2016cez}, both inclusively 
and as differentially~\cite{Dulat:2018bfe,Chen:2021isd,Billis:2021ecs}. The results show a good convergence of the perturbative QCD series, with a 
corresponding decrease of the theoretical uncertainty. However, if comparisons are 
made to fiducial measurements in the diphoton final state, with asymmetric cuts on the transverse momenta of the photons, the predictions show a deterioration 
in the convergence of the predictions and significantly larger scale uncertainties~\cite{Salam:2021tbm}.In addition, the calculation becomes sensitive to very low Higgs 
boson transverse momenta. This can be traced to a significant restriction of the Born phase space for the two photon final state due to the typical form of the
cuts applied to the photon transverse energies, typically quoted as $E^{\gamma 1}_{T} \ge 0.35\, m_{Higgs}, E^{\gamma 2}_{T} \ge 0.25\, m_{Higgs}$. Such a cut was 
initially adopted before the mass of the Higgs boson was known, and thus was meant to be more general (and to insure that the two photons would satisfy the
trigger cuts).

The use of a cut such as defined above leads to larger corrections, and larger uncertainties, for the bulk of the fiducial region (for Born level kinematics),
and a diverging perturbative series. It has been shown that stable perturbative predictions can be restored if the form of the photon transverse energy
cuts is replaced by a product cut: $\sqrt{E^{\gamma 1}_{T} E^{\gamma 2}_{T}}\, \ge f_1\, m_{Higgs}$ and $E^{\gamma 2}_{T}\, \ge \,f_2\, m_{Higgs}$, 
where $f_1$ and
$f_{2}$ are two parameters that need to be determined for the experimental analysis. The goal was to obtain optimal values for $f_1$ and $f_2$ by 

\begin{itemize}
    \item Scan the values of $f_1$ and $f_2$ in the range from 0 to 1. 
    \item Compute the signal efficiency and purity for each of the scanned points. 
    \item Select the product cut value that maximizes the purity and efficiency. 
\end{itemize}

The efficiency and purity for the product cut are defined as:
\vspace{2mm}

Efficiency(product cut) $=\, \frac{\int (ggH\,signal\,after\,applying\,set\,cuts\,+\,product\,cut)}{\int (ggH\,signal\,after\,applying\,set\,cuts)} $
\vspace{2mm}

Purity(product cut) $=\, \frac{\int (ggH\,signal\,after\,applying\,set\,cuts\,+\,product\,cut)}{\int (bkg\,after\,applying\,set\,cuts\,+\,product\,cut)} $

\vspace{2mm}

where the set cuts are defined as:

\begin{itemize}
    \item isPassedPreselection
    \item pass\_trigger
    \item isPassedTriggerMatch
    \item isPassedIsolation
    \item isPassedMassCut
\end{itemize}

The background sample used was mc21a.aMCPy8Eg\_aa\_FxFx\_myy\_90\_175. 


If an optimal point(s) can be found, and if there is no negative impact of using a product cut in the ATLAS analysis, then we will switch to the product cut. 
If such an optimal point(s) can not be obtained (at least equivalent for purity and efficiency), or some other serious difficulty arises,  then the analysis 
would stay with the conventional cuts, and we would live with the decreased theoretical accuracy. 
 
\begin{figure}
\centering
\includegraphics[width=0.75\linewidth]{prod_cut/efficiency_vs_purity_relpt2_025_with_default.pdf}
\caption{This will be replaced by a better quality figure. \label{fig:optimal-product}}
\end{figure}

The relative efficiency and purity for the product cut scan is shown in Fig.~\ref{fig:optimal-product}, comparing the product cut for various values of 
$f_1$ and $f_2$, compared to the traditional relative $E_T$ cut. The optimal $f_1$ and $f_2$ values have been chosen as (0.32,0.25), which result in an 
efficiency and purity very close to those provided by the traditional cut. 
